
% Default to the notebook output style

    


% Inherit from the specified cell style.




    
\documentclass[11pt]{article}

    
    
    \usepackage[T1]{fontenc}
    % Nicer default font (+ math font) than Computer Modern for most use cases
    \usepackage{mathpazo}

    % Basic figure setup, for now with no caption control since it's done
    % automatically by Pandoc (which extracts ![](path) syntax from Markdown).
    \usepackage{graphicx}
    % We will generate all images so they have a width \maxwidth. This means
    % that they will get their normal width if they fit onto the page, but
    % are scaled down if they would overflow the margins.
    \makeatletter
    \def\maxwidth{\ifdim\Gin@nat@width>\linewidth\linewidth
    \else\Gin@nat@width\fi}
    \makeatother
    \let\Oldincludegraphics\includegraphics
    % Set max figure width to be 80% of text width, for now hardcoded.
    \renewcommand{\includegraphics}[1]{\Oldincludegraphics[width=.8\maxwidth]{#1}}
    % Ensure that by default, figures have no caption (until we provide a
    % proper Figure object with a Caption API and a way to capture that
    % in the conversion process - todo).
    \usepackage{caption}
    \DeclareCaptionLabelFormat{nolabel}{}
    \captionsetup{labelformat=nolabel}

    \usepackage{adjustbox} % Used to constrain images to a maximum size 
    \usepackage{xcolor} % Allow colors to be defined
    \usepackage{enumerate} % Needed for markdown enumerations to work
    \usepackage{geometry} % Used to adjust the document margins
    \usepackage{amsmath} % Equations
    \usepackage{amssymb} % Equations
    \usepackage{textcomp} % defines textquotesingle
    % Hack from http://tex.stackexchange.com/a/47451/13684:
    \AtBeginDocument{%
        \def\PYZsq{\textquotesingle}% Upright quotes in Pygmentized code
    }
    \usepackage{upquote} % Upright quotes for verbatim code
    \usepackage{eurosym} % defines \euro
    \usepackage[mathletters]{ucs} % Extended unicode (utf-8) support
    \usepackage[utf8x]{inputenc} % Allow utf-8 characters in the tex document
    \usepackage{fancyvrb} % verbatim replacement that allows latex
    \usepackage{grffile} % extends the file name processing of package graphics 
                         % to support a larger range 
    % The hyperref package gives us a pdf with properly built
    % internal navigation ('pdf bookmarks' for the table of contents,
    % internal cross-reference links, web links for URLs, etc.)
    \usepackage{hyperref}
    \usepackage{longtable} % longtable support required by pandoc >1.10
    \usepackage{booktabs}  % table support for pandoc > 1.12.2
    \usepackage[inline]{enumitem} % IRkernel/repr support (it uses the enumerate* environment)
    \usepackage[normalem]{ulem} % ulem is needed to support strikethroughs (\sout)
                                % normalem makes italics be italics, not underlines
    

    
    
    % Colors for the hyperref package
    \definecolor{urlcolor}{rgb}{0,.145,.698}
    \definecolor{linkcolor}{rgb}{.71,0.21,0.01}
    \definecolor{citecolor}{rgb}{.12,.54,.11}

    % ANSI colors
    \definecolor{ansi-black}{HTML}{3E424D}
    \definecolor{ansi-black-intense}{HTML}{282C36}
    \definecolor{ansi-red}{HTML}{E75C58}
    \definecolor{ansi-red-intense}{HTML}{B22B31}
    \definecolor{ansi-green}{HTML}{00A250}
    \definecolor{ansi-green-intense}{HTML}{007427}
    \definecolor{ansi-yellow}{HTML}{DDB62B}
    \definecolor{ansi-yellow-intense}{HTML}{B27D12}
    \definecolor{ansi-blue}{HTML}{208FFB}
    \definecolor{ansi-blue-intense}{HTML}{0065CA}
    \definecolor{ansi-magenta}{HTML}{D160C4}
    \definecolor{ansi-magenta-intense}{HTML}{A03196}
    \definecolor{ansi-cyan}{HTML}{60C6C8}
    \definecolor{ansi-cyan-intense}{HTML}{258F8F}
    \definecolor{ansi-white}{HTML}{C5C1B4}
    \definecolor{ansi-white-intense}{HTML}{A1A6B2}

    % commands and environments needed by pandoc snippets
    % extracted from the output of `pandoc -s`
    \providecommand{\tightlist}{%
      \setlength{\itemsep}{0pt}\setlength{\parskip}{0pt}}
    \DefineVerbatimEnvironment{Highlighting}{Verbatim}{commandchars=\\\{\}}
    % Add ',fontsize=\small' for more characters per line
    \newenvironment{Shaded}{}{}
    \newcommand{\KeywordTok}[1]{\textcolor[rgb]{0.00,0.44,0.13}{\textbf{{#1}}}}
    \newcommand{\DataTypeTok}[1]{\textcolor[rgb]{0.56,0.13,0.00}{{#1}}}
    \newcommand{\DecValTok}[1]{\textcolor[rgb]{0.25,0.63,0.44}{{#1}}}
    \newcommand{\BaseNTok}[1]{\textcolor[rgb]{0.25,0.63,0.44}{{#1}}}
    \newcommand{\FloatTok}[1]{\textcolor[rgb]{0.25,0.63,0.44}{{#1}}}
    \newcommand{\CharTok}[1]{\textcolor[rgb]{0.25,0.44,0.63}{{#1}}}
    \newcommand{\StringTok}[1]{\textcolor[rgb]{0.25,0.44,0.63}{{#1}}}
    \newcommand{\CommentTok}[1]{\textcolor[rgb]{0.38,0.63,0.69}{\textit{{#1}}}}
    \newcommand{\OtherTok}[1]{\textcolor[rgb]{0.00,0.44,0.13}{{#1}}}
    \newcommand{\AlertTok}[1]{\textcolor[rgb]{1.00,0.00,0.00}{\textbf{{#1}}}}
    \newcommand{\FunctionTok}[1]{\textcolor[rgb]{0.02,0.16,0.49}{{#1}}}
    \newcommand{\RegionMarkerTok}[1]{{#1}}
    \newcommand{\ErrorTok}[1]{\textcolor[rgb]{1.00,0.00,0.00}{\textbf{{#1}}}}
    \newcommand{\NormalTok}[1]{{#1}}
    
    % Additional commands for more recent versions of Pandoc
    \newcommand{\ConstantTok}[1]{\textcolor[rgb]{0.53,0.00,0.00}{{#1}}}
    \newcommand{\SpecialCharTok}[1]{\textcolor[rgb]{0.25,0.44,0.63}{{#1}}}
    \newcommand{\VerbatimStringTok}[1]{\textcolor[rgb]{0.25,0.44,0.63}{{#1}}}
    \newcommand{\SpecialStringTok}[1]{\textcolor[rgb]{0.73,0.40,0.53}{{#1}}}
    \newcommand{\ImportTok}[1]{{#1}}
    \newcommand{\DocumentationTok}[1]{\textcolor[rgb]{0.73,0.13,0.13}{\textit{{#1}}}}
    \newcommand{\AnnotationTok}[1]{\textcolor[rgb]{0.38,0.63,0.69}{\textbf{\textit{{#1}}}}}
    \newcommand{\CommentVarTok}[1]{\textcolor[rgb]{0.38,0.63,0.69}{\textbf{\textit{{#1}}}}}
    \newcommand{\VariableTok}[1]{\textcolor[rgb]{0.10,0.09,0.49}{{#1}}}
    \newcommand{\ControlFlowTok}[1]{\textcolor[rgb]{0.00,0.44,0.13}{\textbf{{#1}}}}
    \newcommand{\OperatorTok}[1]{\textcolor[rgb]{0.40,0.40,0.40}{{#1}}}
    \newcommand{\BuiltInTok}[1]{{#1}}
    \newcommand{\ExtensionTok}[1]{{#1}}
    \newcommand{\PreprocessorTok}[1]{\textcolor[rgb]{0.74,0.48,0.00}{{#1}}}
    \newcommand{\AttributeTok}[1]{\textcolor[rgb]{0.49,0.56,0.16}{{#1}}}
    \newcommand{\InformationTok}[1]{\textcolor[rgb]{0.38,0.63,0.69}{\textbf{\textit{{#1}}}}}
    \newcommand{\WarningTok}[1]{\textcolor[rgb]{0.38,0.63,0.69}{\textbf{\textit{{#1}}}}}
    
    
    % Define a nice break command that doesn't care if a line doesn't already
    % exist.
    \def\br{\hspace*{\fill} \\* }
    % Math Jax compatability definitions
    \def\gt{>}
    \def\lt{<}
    % Document parameters
    \title{Week3\_Honors1-Types}
    
    
    

    % Pygments definitions
    
\makeatletter
\def\PY@reset{\let\PY@it=\relax \let\PY@bf=\relax%
    \let\PY@ul=\relax \let\PY@tc=\relax%
    \let\PY@bc=\relax \let\PY@ff=\relax}
\def\PY@tok#1{\csname PY@tok@#1\endcsname}
\def\PY@toks#1+{\ifx\relax#1\empty\else%
    \PY@tok{#1}\expandafter\PY@toks\fi}
\def\PY@do#1{\PY@bc{\PY@tc{\PY@ul{%
    \PY@it{\PY@bf{\PY@ff{#1}}}}}}}
\def\PY#1#2{\PY@reset\PY@toks#1+\relax+\PY@do{#2}}

\expandafter\def\csname PY@tok@gd\endcsname{\def\PY@tc##1{\textcolor[rgb]{0.63,0.00,0.00}{##1}}}
\expandafter\def\csname PY@tok@gu\endcsname{\let\PY@bf=\textbf\def\PY@tc##1{\textcolor[rgb]{0.50,0.00,0.50}{##1}}}
\expandafter\def\csname PY@tok@gt\endcsname{\def\PY@tc##1{\textcolor[rgb]{0.00,0.27,0.87}{##1}}}
\expandafter\def\csname PY@tok@gs\endcsname{\let\PY@bf=\textbf}
\expandafter\def\csname PY@tok@gr\endcsname{\def\PY@tc##1{\textcolor[rgb]{1.00,0.00,0.00}{##1}}}
\expandafter\def\csname PY@tok@cm\endcsname{\let\PY@it=\textit\def\PY@tc##1{\textcolor[rgb]{0.25,0.50,0.50}{##1}}}
\expandafter\def\csname PY@tok@vg\endcsname{\def\PY@tc##1{\textcolor[rgb]{0.10,0.09,0.49}{##1}}}
\expandafter\def\csname PY@tok@vi\endcsname{\def\PY@tc##1{\textcolor[rgb]{0.10,0.09,0.49}{##1}}}
\expandafter\def\csname PY@tok@vm\endcsname{\def\PY@tc##1{\textcolor[rgb]{0.10,0.09,0.49}{##1}}}
\expandafter\def\csname PY@tok@mh\endcsname{\def\PY@tc##1{\textcolor[rgb]{0.40,0.40,0.40}{##1}}}
\expandafter\def\csname PY@tok@cs\endcsname{\let\PY@it=\textit\def\PY@tc##1{\textcolor[rgb]{0.25,0.50,0.50}{##1}}}
\expandafter\def\csname PY@tok@ge\endcsname{\let\PY@it=\textit}
\expandafter\def\csname PY@tok@vc\endcsname{\def\PY@tc##1{\textcolor[rgb]{0.10,0.09,0.49}{##1}}}
\expandafter\def\csname PY@tok@il\endcsname{\def\PY@tc##1{\textcolor[rgb]{0.40,0.40,0.40}{##1}}}
\expandafter\def\csname PY@tok@go\endcsname{\def\PY@tc##1{\textcolor[rgb]{0.53,0.53,0.53}{##1}}}
\expandafter\def\csname PY@tok@cp\endcsname{\def\PY@tc##1{\textcolor[rgb]{0.74,0.48,0.00}{##1}}}
\expandafter\def\csname PY@tok@gi\endcsname{\def\PY@tc##1{\textcolor[rgb]{0.00,0.63,0.00}{##1}}}
\expandafter\def\csname PY@tok@gh\endcsname{\let\PY@bf=\textbf\def\PY@tc##1{\textcolor[rgb]{0.00,0.00,0.50}{##1}}}
\expandafter\def\csname PY@tok@ni\endcsname{\let\PY@bf=\textbf\def\PY@tc##1{\textcolor[rgb]{0.60,0.60,0.60}{##1}}}
\expandafter\def\csname PY@tok@nl\endcsname{\def\PY@tc##1{\textcolor[rgb]{0.63,0.63,0.00}{##1}}}
\expandafter\def\csname PY@tok@nn\endcsname{\let\PY@bf=\textbf\def\PY@tc##1{\textcolor[rgb]{0.00,0.00,1.00}{##1}}}
\expandafter\def\csname PY@tok@no\endcsname{\def\PY@tc##1{\textcolor[rgb]{0.53,0.00,0.00}{##1}}}
\expandafter\def\csname PY@tok@na\endcsname{\def\PY@tc##1{\textcolor[rgb]{0.49,0.56,0.16}{##1}}}
\expandafter\def\csname PY@tok@nb\endcsname{\def\PY@tc##1{\textcolor[rgb]{0.00,0.50,0.00}{##1}}}
\expandafter\def\csname PY@tok@nc\endcsname{\let\PY@bf=\textbf\def\PY@tc##1{\textcolor[rgb]{0.00,0.00,1.00}{##1}}}
\expandafter\def\csname PY@tok@nd\endcsname{\def\PY@tc##1{\textcolor[rgb]{0.67,0.13,1.00}{##1}}}
\expandafter\def\csname PY@tok@ne\endcsname{\let\PY@bf=\textbf\def\PY@tc##1{\textcolor[rgb]{0.82,0.25,0.23}{##1}}}
\expandafter\def\csname PY@tok@nf\endcsname{\def\PY@tc##1{\textcolor[rgb]{0.00,0.00,1.00}{##1}}}
\expandafter\def\csname PY@tok@si\endcsname{\let\PY@bf=\textbf\def\PY@tc##1{\textcolor[rgb]{0.73,0.40,0.53}{##1}}}
\expandafter\def\csname PY@tok@s2\endcsname{\def\PY@tc##1{\textcolor[rgb]{0.73,0.13,0.13}{##1}}}
\expandafter\def\csname PY@tok@nt\endcsname{\let\PY@bf=\textbf\def\PY@tc##1{\textcolor[rgb]{0.00,0.50,0.00}{##1}}}
\expandafter\def\csname PY@tok@nv\endcsname{\def\PY@tc##1{\textcolor[rgb]{0.10,0.09,0.49}{##1}}}
\expandafter\def\csname PY@tok@s1\endcsname{\def\PY@tc##1{\textcolor[rgb]{0.73,0.13,0.13}{##1}}}
\expandafter\def\csname PY@tok@dl\endcsname{\def\PY@tc##1{\textcolor[rgb]{0.73,0.13,0.13}{##1}}}
\expandafter\def\csname PY@tok@ch\endcsname{\let\PY@it=\textit\def\PY@tc##1{\textcolor[rgb]{0.25,0.50,0.50}{##1}}}
\expandafter\def\csname PY@tok@m\endcsname{\def\PY@tc##1{\textcolor[rgb]{0.40,0.40,0.40}{##1}}}
\expandafter\def\csname PY@tok@gp\endcsname{\let\PY@bf=\textbf\def\PY@tc##1{\textcolor[rgb]{0.00,0.00,0.50}{##1}}}
\expandafter\def\csname PY@tok@sh\endcsname{\def\PY@tc##1{\textcolor[rgb]{0.73,0.13,0.13}{##1}}}
\expandafter\def\csname PY@tok@ow\endcsname{\let\PY@bf=\textbf\def\PY@tc##1{\textcolor[rgb]{0.67,0.13,1.00}{##1}}}
\expandafter\def\csname PY@tok@sx\endcsname{\def\PY@tc##1{\textcolor[rgb]{0.00,0.50,0.00}{##1}}}
\expandafter\def\csname PY@tok@bp\endcsname{\def\PY@tc##1{\textcolor[rgb]{0.00,0.50,0.00}{##1}}}
\expandafter\def\csname PY@tok@c1\endcsname{\let\PY@it=\textit\def\PY@tc##1{\textcolor[rgb]{0.25,0.50,0.50}{##1}}}
\expandafter\def\csname PY@tok@fm\endcsname{\def\PY@tc##1{\textcolor[rgb]{0.00,0.00,1.00}{##1}}}
\expandafter\def\csname PY@tok@o\endcsname{\def\PY@tc##1{\textcolor[rgb]{0.40,0.40,0.40}{##1}}}
\expandafter\def\csname PY@tok@kc\endcsname{\let\PY@bf=\textbf\def\PY@tc##1{\textcolor[rgb]{0.00,0.50,0.00}{##1}}}
\expandafter\def\csname PY@tok@c\endcsname{\let\PY@it=\textit\def\PY@tc##1{\textcolor[rgb]{0.25,0.50,0.50}{##1}}}
\expandafter\def\csname PY@tok@mf\endcsname{\def\PY@tc##1{\textcolor[rgb]{0.40,0.40,0.40}{##1}}}
\expandafter\def\csname PY@tok@err\endcsname{\def\PY@bc##1{\setlength{\fboxsep}{0pt}\fcolorbox[rgb]{1.00,0.00,0.00}{1,1,1}{\strut ##1}}}
\expandafter\def\csname PY@tok@mb\endcsname{\def\PY@tc##1{\textcolor[rgb]{0.40,0.40,0.40}{##1}}}
\expandafter\def\csname PY@tok@ss\endcsname{\def\PY@tc##1{\textcolor[rgb]{0.10,0.09,0.49}{##1}}}
\expandafter\def\csname PY@tok@sr\endcsname{\def\PY@tc##1{\textcolor[rgb]{0.73,0.40,0.53}{##1}}}
\expandafter\def\csname PY@tok@mo\endcsname{\def\PY@tc##1{\textcolor[rgb]{0.40,0.40,0.40}{##1}}}
\expandafter\def\csname PY@tok@kd\endcsname{\let\PY@bf=\textbf\def\PY@tc##1{\textcolor[rgb]{0.00,0.50,0.00}{##1}}}
\expandafter\def\csname PY@tok@mi\endcsname{\def\PY@tc##1{\textcolor[rgb]{0.40,0.40,0.40}{##1}}}
\expandafter\def\csname PY@tok@kn\endcsname{\let\PY@bf=\textbf\def\PY@tc##1{\textcolor[rgb]{0.00,0.50,0.00}{##1}}}
\expandafter\def\csname PY@tok@cpf\endcsname{\let\PY@it=\textit\def\PY@tc##1{\textcolor[rgb]{0.25,0.50,0.50}{##1}}}
\expandafter\def\csname PY@tok@kr\endcsname{\let\PY@bf=\textbf\def\PY@tc##1{\textcolor[rgb]{0.00,0.50,0.00}{##1}}}
\expandafter\def\csname PY@tok@s\endcsname{\def\PY@tc##1{\textcolor[rgb]{0.73,0.13,0.13}{##1}}}
\expandafter\def\csname PY@tok@kp\endcsname{\def\PY@tc##1{\textcolor[rgb]{0.00,0.50,0.00}{##1}}}
\expandafter\def\csname PY@tok@w\endcsname{\def\PY@tc##1{\textcolor[rgb]{0.73,0.73,0.73}{##1}}}
\expandafter\def\csname PY@tok@kt\endcsname{\def\PY@tc##1{\textcolor[rgb]{0.69,0.00,0.25}{##1}}}
\expandafter\def\csname PY@tok@sc\endcsname{\def\PY@tc##1{\textcolor[rgb]{0.73,0.13,0.13}{##1}}}
\expandafter\def\csname PY@tok@sb\endcsname{\def\PY@tc##1{\textcolor[rgb]{0.73,0.13,0.13}{##1}}}
\expandafter\def\csname PY@tok@sa\endcsname{\def\PY@tc##1{\textcolor[rgb]{0.73,0.13,0.13}{##1}}}
\expandafter\def\csname PY@tok@k\endcsname{\let\PY@bf=\textbf\def\PY@tc##1{\textcolor[rgb]{0.00,0.50,0.00}{##1}}}
\expandafter\def\csname PY@tok@se\endcsname{\let\PY@bf=\textbf\def\PY@tc##1{\textcolor[rgb]{0.73,0.40,0.13}{##1}}}
\expandafter\def\csname PY@tok@sd\endcsname{\let\PY@it=\textit\def\PY@tc##1{\textcolor[rgb]{0.73,0.13,0.13}{##1}}}

\def\PYZbs{\char`\\}
\def\PYZus{\char`\_}
\def\PYZob{\char`\{}
\def\PYZcb{\char`\}}
\def\PYZca{\char`\^}
\def\PYZam{\char`\&}
\def\PYZlt{\char`\<}
\def\PYZgt{\char`\>}
\def\PYZsh{\char`\#}
\def\PYZpc{\char`\%}
\def\PYZdl{\char`\$}
\def\PYZhy{\char`\-}
\def\PYZsq{\char`\'}
\def\PYZdq{\char`\"}
\def\PYZti{\char`\~}
% for compatibility with earlier versions
\def\PYZat{@}
\def\PYZlb{[}
\def\PYZrb{]}
\makeatother


    % Exact colors from NB
    \definecolor{incolor}{rgb}{0.0, 0.0, 0.5}
    \definecolor{outcolor}{rgb}{0.545, 0.0, 0.0}



    
    % Prevent overflowing lines due to hard-to-break entities
    \sloppy 
    % Setup hyperref package
    \hypersetup{
      breaklinks=true,  % so long urls are correctly broken across lines
      colorlinks=true,
      urlcolor=urlcolor,
      linkcolor=linkcolor,
      citecolor=citecolor,
      }
    % Slightly bigger margins than the latex defaults
    
    \geometry{verbose,tmargin=1in,bmargin=1in,lmargin=1in,rmargin=1in}
    
    

    \begin{document}
    
    
    \maketitle
    
    

    
    \begin{Verbatim}[commandchars=\\\{\}]
{\color{incolor}In [{\color{incolor}1}]:} \PY{c}{\PYZsh{} Setting up a custom stylesheet in IJulia}
        \PY{n}{file} \PY{o}{=} \PY{n}{open}\PY{p}{(}\PY{l+s}{\PYZdq{}}\PY{l+s}{s}\PY{l+s}{t}\PY{l+s}{y}\PY{l+s}{l}\PY{l+s}{e}\PY{l+s}{.}\PY{l+s}{c}\PY{l+s}{s}\PY{l+s}{s}\PY{l+s}{\PYZdq{}}\PY{p}{)} \PY{c}{\PYZsh{} A .css file in the same folder as this notebook file}
        \PY{n}{styl} \PY{o}{=} \PY{n}{readstring}\PY{p}{(}\PY{n}{file}\PY{p}{)} \PY{c}{\PYZsh{} Read the file}
        \PY{k+kt}{HTML}\PY{p}{(}\PY{l+s}{\PYZdq{}}\PY{l+s+si}{\PYZdl{}styl}\PY{l+s}{\PYZdq{}}\PY{p}{)} \PY{c}{\PYZsh{} Output as HTML}
\end{Verbatim}


\begin{Verbatim}[commandchars=\\\{\}]
{\color{outcolor}Out[{\color{outcolor}1}]:} HTML\{String\}("<link href='http://fonts.googleapis.com/css?family=Alegreya+Sans:100,300,400,500,700,800,900,100italic,300italic,400italic,500italic,700italic,800italic,900italic' rel='stylesheet' type='text/css'>\textbackslash{}r\textbackslash{}n<link href='http://fonts.googleapis.com/css?family=Arvo:400,700,400italic' rel='stylesheet' type='text/css'>\textbackslash{}r\textbackslash{}n<link href='http://fonts.googleapis.com/css?family=PT+Mono' rel='stylesheet' type='text/css'>\textbackslash{}r\textbackslash{}n<link href='http://fonts.googleapis.com/css?family=Shadows+Into+Light' rel='stylesheet' type='text/css'>\textbackslash{}r\textbackslash{}n<link href='http://fonts.googleapis.com/css?family=Philosopher:400,700,400italic,700italic' rel='stylesheet' type='text/css'>\textbackslash{}r\textbackslash{}n\textbackslash{}r\textbackslash{}n<style>\textbackslash{}r\textbackslash{}n\textbackslash{}r\textbackslash{}n@font-face \{\textbackslash{}r\textbackslash{}n    font-family: \textbackslash{}"Computer Modern\textbackslash{}";\textbackslash{}r\textbackslash{}n    src: url('http://mirrors.ctan.org/fonts/cm-unicode/fonts/otf/cmunss.otf');\textbackslash{}r\textbackslash{}n\}\textbackslash{}r\textbackslash{}n\textbackslash{}r\textbackslash{}n\#notebook\_panel \{ /* main background */\textbackslash{}r\textbackslash{}n    background: \#ddd;\textbackslash{}r\textbackslash{}n    color: \#000000;\textbackslash{}r\textbackslash{}n\}\textbackslash{}r\textbackslash{}n\textbackslash{}r\textbackslash{}n\textbackslash{}r\textbackslash{}n\textbackslash{}r\textbackslash{}n/* Formatting for header cells */\textbackslash{}r\textbackslash{}n.text\_cell\_render h1 \{\textbackslash{}r\textbackslash{}n    font-family: 'Philosopher', sans-serif;\textbackslash{}r\textbackslash{}n    font-weight: 400;\textbackslash{}r\textbackslash{}n    font-size: 2.2em;\textbackslash{}r\textbackslash{}n    line-height: 100\%;\textbackslash{}r\textbackslash{}n    color: rgb(0, 80, 120);\textbackslash{}r\textbackslash{}n    margin-bottom: 0.1em;\textbackslash{}r\textbackslash{}n    margin-top: 0.1em;\textbackslash{}r\textbackslash{}n    display: block;\textbackslash{}r\textbackslash{}n\}\textbackslash{}t\textbackslash{}r\textbackslash{}n.text\_cell\_render h2 \{\textbackslash{}r\textbackslash{}n    font-family: 'Philosopher', serif;\textbackslash{}r\textbackslash{}n    font-weight: 400;\textbackslash{}r\textbackslash{}n    font-size: 1.9em;\textbackslash{}r\textbackslash{}n    line-height: 100\%;\textbackslash{}r\textbackslash{}n    color: rgb(200,100,0);\textbackslash{}r\textbackslash{}n    margin-bottom: 0.1em;\textbackslash{}r\textbackslash{}n    margin-top: 0.1em;\textbackslash{}r\textbackslash{}n    display: block;\textbackslash{}r\textbackslash{}n\}\textbackslash{}t\textbackslash{}r\textbackslash{}n\textbackslash{}r\textbackslash{}n.text\_cell\_render h3 \{\textbackslash{}r\textbackslash{}n    font-family: 'Philosopher', serif;\textbackslash{}r\textbackslash{}n    margin-top:12px;\textbackslash{}r\textbackslash{}n    margin-bottom: 3px;\textbackslash{}r\textbackslash{}n    font-style: italic;\textbackslash{}r\textbackslash{}n    color: rgb(94,127,192);\textbackslash{}r\textbackslash{}n\}\textbackslash{}r\textbackslash{}n\textbackslash{}r\textbackslash{}n.text\_cell\_render h4 \{\textbackslash{}r\textbackslash{}n    font-family: 'Philosopher', serif;\textbackslash{}r\textbackslash{}n\}\textbackslash{}r\textbackslash{}n\textbackslash{}r\textbackslash{}n.text\_cell\_render h5 \{\textbackslash{}r\textbackslash{}n    font-family: 'Alegreya Sans', sans-serif;\textbackslash{}r\textbackslash{}n    font-weight: 300;\textbackslash{}r\textbackslash{}n    font-size: 16pt;\textbackslash{}r\textbackslash{}n    color: grey;\textbackslash{}r\textbackslash{}n    font-style: italic;\textbackslash{}r\textbackslash{}n    margin-bottom: .1em;\textbackslash{}r\textbackslash{}n    margin-top: 0.1em;\textbackslash{}r\textbackslash{}n    display: block;\textbackslash{}r\textbackslash{}n\}\textbackslash{}r\textbackslash{}n\textbackslash{}r\textbackslash{}n.text\_cell\_render h6 \{\textbackslash{}r\textbackslash{}n    font-family: 'PT Mono', sans-serif;\textbackslash{}r\textbackslash{}n    font-weight: 300;\textbackslash{}r\textbackslash{}n    font-size: 10pt;\textbackslash{}r\textbackslash{}n    color: grey;\textbackslash{}r\textbackslash{}n    margin-bottom: 1px;\textbackslash{}r\textbackslash{}n    margin-top: 1px;\textbackslash{}r\textbackslash{}n\}\textbackslash{}r\textbackslash{}n\textbackslash{}r\textbackslash{}n.CodeMirror\{\textbackslash{}r\textbackslash{}n        font-family: \textbackslash{}"PT Mono\textbackslash{}";\textbackslash{}r\textbackslash{}n        font-size: 100\%;\textbackslash{}r\textbackslash{}n\}\textbackslash{}r\textbackslash{}n\textbackslash{}r\textbackslash{}n</style>\textbackslash{}r\textbackslash{}n\textbackslash{}r\textbackslash{}n")
\end{Verbatim}
            
    \section{Types}\label{types}

    In this lesson

    \begin{itemize}
\tightlist
\item
  Section \ref{introduction}
\item
  Section \ref{importing-the-packages-for-this-lesson}
\item
  Section \ref{outcomes}
\item
  Section \ref{the-type-system-in-julia}
\item
  Section \ref{type-creation}
\item
  Section \ref{convertion-and-promotion}
\item
  Section \ref{parametrizing-a-type}
\item
  Section \ref{the-equality-of-values}
\item
  Section \ref{defining-methods-for-functions-that-will-use-user-types}
\item
  Section \ref{constraining-field-values}
\item
  Section \ref{more-complex-parameters}
\item
  Section \ref{screen-output-of-a-user-defined-type}
\end{itemize}

    Introduction

    A computer variable, which is a space in memory, holds values of
different types, i.e. integers, floating point values, and strings. In
some languages the type of the value to be held inside of a variable
must be explicitely declared. These languages are termed
\emph{statically typed}. In \emph{dynamically typed} languages nothing
is known about the type of the value held inside the variable until
runtime. Being able to write code operating on different types is termed
\emph{polymorphism}.

    Julia is a dynamically typed language, yet it is possible to declare a
type for values as well. Declaring a type allows for code that is clear
to understand. As an assertion it can help to confirm that your code is
working as expected. It can also allow for faster code execution by
providing the compiler with extra information.

    Section \ref{in-this-lesson}

    Outcomes

    After successfully completing this lecture, you will be able to:

\begin{itemize}
\tightlist
\item
  Understand the Julia type system
\item
  Create your own user-defined types
\item
  Parametrize your types
\item
  Overload methods for Julia functions so that they can use your types
\end{itemize}

    Section \ref{in-this-lesson}

    The type system in Julia

    Julia holds a type hierarchy that flows like the branches of a tree.
Right at the top we have a type called \texttt{Any}. All types are
subtypes of this type. Right at the final tip of the branches we have
concrete types. They can hold values. Supertypes of these concrete types
are called abtsract types and they cannot hold values, i.e. we cannot
create an instance of an abstract type.

    We can use Julia to see if types are subtypes of a supertype.

    \begin{Verbatim}[commandchars=\\\{\}]
{\color{incolor}In [{\color{incolor}2}]:} \PY{c}{\PYZsh{} Is Number a subtype of Any?}
        \PY{k+kt}{Number} \PY{o}{\PYZlt{}:} \PY{k+kt}{Any}
\end{Verbatim}


\begin{Verbatim}[commandchars=\\\{\}]
{\color{outcolor}Out[{\color{outcolor}2}]:} true
\end{Verbatim}
            
    \begin{Verbatim}[commandchars=\\\{\}]
{\color{incolor}In [{\color{incolor}3}]:} \PY{c}{\PYZsh{} Is Float 64 a subtype of AbstractFloat?}
        \PY{k+kt}{Float64} \PY{o}{\PYZlt{}:} \PY{k+kt}{AbstractFloat}
\end{Verbatim}


\begin{Verbatim}[commandchars=\\\{\}]
{\color{outcolor}Out[{\color{outcolor}3}]:} true
\end{Verbatim}
            
    \begin{Verbatim}[commandchars=\\\{\}]
{\color{incolor}In [{\color{incolor}4}]:} \PY{c}{\PYZsh{} The subtypes of Any}
        \PY{n}{subtypes}\PY{p}{(}\PY{k+kt}{Any}\PY{p}{)}
\end{Verbatim}


\begin{Verbatim}[commandchars=\\\{\}]
{\color{outcolor}Out[{\color{outcolor}4}]:} 295-element Array\{Union\{DataType, UnionAll\},1\}:
         AbstractArray              
         AbstractChannel            
         AbstractRNG                
         AbstractSerializer         
         AbstractSet                
         AbstractString             
         Any                        
         Associative                
         Base.AbstractCartesianIndex
         Base.AbstractCmd           
         Base.AsyncCollector        
         Base.AsyncCollectorState   
         Base.AsyncCondition        
         ⋮                          
         TypeVar                    
         UniformScaling             
         Val                        
         Vararg                     
         VecElement                 
         VersionNumber              
         Void                       
         WeakRef                    
         WorkerConfig               
         ZMQ.Context                
         ZMQ.MsgPadding             
         ZMQ.Socket                 
\end{Verbatim}
            
    \begin{Verbatim}[commandchars=\\\{\}]
{\color{incolor}In [{\color{incolor}5}]:} \PY{c}{\PYZsh{} Subtypes of AbstractString}
        \PY{n}{subtypes}\PY{p}{(}\PY{k+kt}{AbstractString}\PY{p}{)}
\end{Verbatim}


\begin{Verbatim}[commandchars=\\\{\}]
{\color{outcolor}Out[{\color{outcolor}5}]:} 6-element Array\{Union\{DataType, UnionAll\},1\}:
         Base.SubstitutionString
         Base.Test.GenericString
         DirectIndexString      
         RevString              
         String                 
         SubString              
\end{Verbatim}
            
    \begin{Verbatim}[commandchars=\\\{\}]
{\color{incolor}In [{\color{incolor}6}]:} \PY{c}{\PYZsh{} Subtypes of Number}
        \PY{n}{subtypes}\PY{p}{(}\PY{k+kt}{Number}\PY{p}{)}
\end{Verbatim}


\begin{Verbatim}[commandchars=\\\{\}]
{\color{outcolor}Out[{\color{outcolor}6}]:} 2-element Array\{Union\{DataType, UnionAll\},1\}:
         Complex
         Real   
\end{Verbatim}
            
    \begin{Verbatim}[commandchars=\\\{\}]
{\color{incolor}In [{\color{incolor}7}]:} \PY{c}{\PYZsh{} Subtypes of Real}
        \PY{n}{subtypes}\PY{p}{(}\PY{k+kt}{Real}\PY{p}{)}
\end{Verbatim}


\begin{Verbatim}[commandchars=\\\{\}]
{\color{outcolor}Out[{\color{outcolor}7}]:} 4-element Array\{Union\{DataType, UnionAll\},1\}:
         AbstractFloat
         Integer      
         Irrational   
         Rational     
\end{Verbatim}
            
    \begin{Verbatim}[commandchars=\\\{\}]
{\color{incolor}In [{\color{incolor}8}]:} \PY{c}{\PYZsh{} Subtypes of AbstractFloat}
        \PY{n}{subtypes}\PY{p}{(}\PY{k+kt}{AbstractFloat}\PY{p}{)}
\end{Verbatim}


\begin{Verbatim}[commandchars=\\\{\}]
{\color{outcolor}Out[{\color{outcolor}8}]:} 4-element Array\{Union\{DataType, UnionAll\},1\}:
         BigFloat
         Float16 
         Float32 
         Float64 
\end{Verbatim}
            
    \begin{Verbatim}[commandchars=\\\{\}]
{\color{incolor}In [{\color{incolor}9}]:} \PY{c}{\PYZsh{} Subtypes of Integer}
        \PY{n}{subtypes}\PY{p}{(}\PY{k+kt}{Integer}\PY{p}{)}
\end{Verbatim}


\begin{Verbatim}[commandchars=\\\{\}]
{\color{outcolor}Out[{\color{outcolor}9}]:} 4-element Array\{Union\{DataType, UnionAll\},1\}:
         BigInt  
         Bool    
         Signed  
         Unsigned
\end{Verbatim}
            
    \begin{Verbatim}[commandchars=\\\{\}]
{\color{incolor}In [{\color{incolor}10}]:} \PY{c}{\PYZsh{} Subtypes of Signed}
         \PY{n}{subtypes}\PY{p}{(}\PY{k+kt}{Signed}\PY{p}{)}
\end{Verbatim}


\begin{Verbatim}[commandchars=\\\{\}]
{\color{outcolor}Out[{\color{outcolor}10}]:} 5-element Array\{Union\{DataType, UnionAll\},1\}:
          Int128
          Int16 
          Int32 
          Int64 
          Int8  
\end{Verbatim}
            
    Section \ref{in-this-lesson}

    Declaring a type

    A type is declared by the double colon, \texttt{::}, sign. To the left
we place the value (or placeholder for a variable, i.e. a variable name)
and to the right the actual type. In the example below we want to
express the fact that the value \$ 2 + 2 \$ is an instance of a 64-bit
integer.

    \begin{Verbatim}[commandchars=\\\{\}]
{\color{incolor}In [{\color{incolor}11}]:} \PY{p}{(}\PY{l+m+mi}{2} \PY{o}{+} \PY{l+m+mi}{2}\PY{p}{)}\PY{o}{::}\PY{k+kt}{Int64}
\end{Verbatim}


\begin{Verbatim}[commandchars=\\\{\}]
{\color{outcolor}Out[{\color{outcolor}11}]:} 4
\end{Verbatim}
            
    If we typed \texttt{(2\ +\ 2)::Float64} we would get the following
error:

\begin{verbatim}
LoadError: TypeError: typeassert: expected Float64, got Int64
while loading In[3], in expression starting on line 1
\end{verbatim}

We used the declaration of a type as an assertion, which allowed us to
see that there was something wrong with our code. We can imagine a
program where the \texttt{+()} (or a more complicated user-defined)
function is called and we need the arguments to be of a certain type. An
error such as the one above can give us information about what went
wrong.

    Declaring a type of a local variable (inside of a function), we state
that the type should always remain the same. This is more like what
would happen in a statically typed language. It is really helpful if we
want an error to be thrown should the type of a variable be changed by
another part of our code. This can lead to type instability. It can
really impact the speed of execution.

    \begin{Verbatim}[commandchars=\\\{\}]
{\color{incolor}In [{\color{incolor}12}]:} \PY{c}{\PYZsh{} Creating a function with a local variable}
         \PY{k}{function} \PY{n}{static\PYZus{}local\PYZus{}variable}\PY{p}{(}\PY{p}{)}
             \PY{n}{v}\PY{o}{::}\PY{k+kt}{Int16} \PY{o}{=} \PY{l+m+mi}{42}
             \PY{k}{return} \PY{n}{v}
         \PY{k}{end}
\end{Verbatim}


\begin{Verbatim}[commandchars=\\\{\}]
{\color{outcolor}Out[{\color{outcolor}12}]:} static\_local\_variable (generic function with 1 method)
\end{Verbatim}
            
    \begin{Verbatim}[commandchars=\\\{\}]
{\color{incolor}In [{\color{incolor}13}]:} \PY{c}{\PYZsh{} Calling the function}
         \PY{n}{static\PYZus{}local\PYZus{}variable}\PY{p}{(}\PY{p}{)}
\end{Verbatim}


\begin{Verbatim}[commandchars=\\\{\}]
{\color{outcolor}Out[{\color{outcolor}13}]:} 42
\end{Verbatim}
            
    \begin{Verbatim}[commandchars=\\\{\}]
{\color{incolor}In [{\color{incolor}14}]:} \PY{c}{\PYZsh{} Checking the type of the answer just give}
         \PY{n}{typeof}\PY{p}{(}\PY{n}{ans}\PY{p}{)}
\end{Verbatim}


\begin{Verbatim}[commandchars=\\\{\}]
{\color{outcolor}Out[{\color{outcolor}14}]:} Int16
\end{Verbatim}
            
    Remember that \texttt{v} is local to the function. If we try and look at
it value by typing \texttt{v}, we would get the following error:

\begin{verbatim}
LoadError: UndefVarError: v not defined
while loading In[7], in expression starting on line 1
\end{verbatim}

    Now that we know something about declaring a type, let's look at
creating our own types.

    Section \ref{in-this-lesson}

    Type creation

    As mentioned, we can create our own types. Consider a Cartesian
coordinate system along two perpendicular axes, say \$ x \$ and \$ y \$.
A vector in the plane can be represented as a type. The keyword we use
to create a type is \texttt{type}. If we want instances of our type to
be immutable, we use the keyword \texttt{immutable}.

    \begin{Verbatim}[commandchars=\\\{\}]
{\color{incolor}In [{\color{incolor}15}]:} \PY{c}{\PYZsh{} Creating a concrete type called Vector\PYZus{}2D}
         \PY{k}{type} \PY{n}{Vector\PYZus{}2D}
             \PY{n}{x}\PY{o}{::}\PY{k+kt}{Float64} \PY{c}{\PYZsh{} x is a fieldname of the type and has an optional type}
             \PY{n}{y}\PY{o}{::}\PY{k+kt}{Float64} \PY{c}{\PYZsh{} y is a fieldname of the type and has an optional type}
         \PY{k}{end}
\end{Verbatim}


    This is actually a composite type, since we have fields. For a type that
is non-composite we can imagine a simple wrapper around an already
defined type, such as we do below.

    \begin{Verbatim}[commandchars=\\\{\}]
{\color{incolor}In [{\color{incolor}16}]:} \PY{c}{\PYZsh{} A non\PYZhy{}composite type}
         \PY{k}{type} \PY{n}{NonComposite}
             \PY{n}{x}\PY{o}{::}\PY{k+kt}{Float64}
         \PY{k}{end}
\end{Verbatim}


    \begin{Verbatim}[commandchars=\\\{\}]
{\color{incolor}In [{\color{incolor}17}]:} \PY{n}{my\PYZus{}non\PYZus{}composite} \PY{o}{=} \PY{n}{NonComposite}\PY{p}{(}\PY{l+m+mi}{42}\PY{p}{)}
\end{Verbatim}


\begin{Verbatim}[commandchars=\\\{\}]
{\color{outcolor}Out[{\color{outcolor}17}]:} NonComposite(42.0)
\end{Verbatim}
            
    \begin{Verbatim}[commandchars=\\\{\}]
{\color{incolor}In [{\color{incolor}18}]:} \PY{c}{\PYZsh{} Type}
         \PY{n}{typeof}\PY{p}{(}\PY{n}{ans}\PY{p}{)}
\end{Verbatim}


\begin{Verbatim}[commandchars=\\\{\}]
{\color{outcolor}Out[{\color{outcolor}18}]:} NonComposite
\end{Verbatim}
            
    Back to the more exciting composite types. We can now instantiate the
concrete type \texttt{Vector\_2D}.

    \begin{Verbatim}[commandchars=\\\{\}]
{\color{incolor}In [{\color{incolor}19}]:} \PY{n}{vector\PYZus{}1} \PY{o}{=} \PY{n}{Vector\PYZus{}2D}\PY{p}{(}\PY{l+m+mi}{2}\PY{p}{,} \PY{l+m+mi}{2}\PY{p}{)}
\end{Verbatim}


\begin{Verbatim}[commandchars=\\\{\}]
{\color{outcolor}Out[{\color{outcolor}19}]:} Vector\_2D(2.0, 2.0)
\end{Verbatim}
            
    \begin{Verbatim}[commandchars=\\\{\}]
{\color{incolor}In [{\color{incolor}20}]:} \PY{c}{\PYZsh{} The type of vector\PYZus{}1}
         \PY{n}{typeof}\PY{p}{(}\PY{n}{vector\PYZus{}1}\PY{p}{)}
\end{Verbatim}


\begin{Verbatim}[commandchars=\\\{\}]
{\color{outcolor}Out[{\color{outcolor}20}]:} Vector\_2D
\end{Verbatim}
            
    Notice how we get floating point values even though we gave two integer
values. The \texttt{convert()} functions was created to change allowable
values to 64-bit floating point values.

    Also notice that it looks like we called a function when we typed
\texttt{Vector\_2D(2,\ 2)}. When we define a type, constructors are
created. They allow us to create an instance of that type (sometimes
referred to as an \emph{object} of that type).

    As with functions, we can access the methods that were created with the
type.

    \begin{Verbatim}[commandchars=\\\{\}]
{\color{incolor}In [{\color{incolor}21}]:} \PY{n}{methods}\PY{p}{(}\PY{n}{Vector\PYZus{}2D}\PY{p}{)}
\end{Verbatim}


\begin{Verbatim}[commandchars=\\\{\}]
{\color{outcolor}Out[{\color{outcolor}21}]:} \# 3 methods for generic function "(::Type)":
         Vector\_2D(x::Float64, y::Float64) in Main at In[15]:3
         Vector\_2D(x, y) in Main at In[15]:3
         (::Type\{T\})(arg) where T in Base at sysimg.jl:24
\end{Verbatim}
            
    We can also access the fieldnames and their values. They are mutable,
i.e. we can pass new values to them.

    \begin{Verbatim}[commandchars=\\\{\}]
{\color{incolor}In [{\color{incolor}22}]:} \PY{c}{\PYZsh{} The available names (fields, fieldnames)}
         \PY{c}{\PYZsh{} Note that they are of type Symbol}
         \PY{n}{fieldnames}\PY{p}{(}\PY{n}{Vector\PYZus{}2D}\PY{p}{)}
\end{Verbatim}


\begin{Verbatim}[commandchars=\\\{\}]
{\color{outcolor}Out[{\color{outcolor}22}]:} 2-element Array\{Symbol,1\}:
          :x
          :y
\end{Verbatim}
            
    \begin{Verbatim}[commandchars=\\\{\}]
{\color{incolor}In [{\color{incolor}23}]:} \PY{c}{\PYZsh{} Getting the value of the :x field}
         \PY{n}{vector\PYZus{}1}\PY{o}{.}\PY{n}{x}
\end{Verbatim}


\begin{Verbatim}[commandchars=\\\{\}]
{\color{outcolor}Out[{\color{outcolor}23}]:} 2.0
\end{Verbatim}
            
    \begin{Verbatim}[commandchars=\\\{\}]
{\color{incolor}In [{\color{incolor}24}]:} \PY{c}{\PYZsh{} Alternative syntax using the field\PYZsq{}s symbol representation}
         \PY{n}{getfield}\PY{p}{(}\PY{n}{vector\PYZus{}1}\PY{p}{,} \PY{o}{:}\PY{n}{x}\PY{p}{)}
\end{Verbatim}


\begin{Verbatim}[commandchars=\\\{\}]
{\color{outcolor}Out[{\color{outcolor}24}]:} 2.0
\end{Verbatim}
            
    \begin{Verbatim}[commandchars=\\\{\}]
{\color{incolor}In [{\color{incolor}25}]:} \PY{c}{\PYZsh{} Another alternative notation using the index number of the fields}
         \PY{n}{getfield}\PY{p}{(}\PY{n}{vector\PYZus{}1}\PY{p}{,} \PY{l+m+mi}{1}\PY{p}{)}
\end{Verbatim}


\begin{Verbatim}[commandchars=\\\{\}]
{\color{outcolor}Out[{\color{outcolor}25}]:} 2.0
\end{Verbatim}
            
    \begin{Verbatim}[commandchars=\\\{\}]
{\color{incolor}In [{\color{incolor}26}]:} \PY{n}{vector\PYZus{}1}\PY{o}{.}\PY{n}{x} \PY{o}{=} \PY{l+m+mi}{3}
\end{Verbatim}


\begin{Verbatim}[commandchars=\\\{\}]
{\color{outcolor}Out[{\color{outcolor}26}]:} 3
\end{Verbatim}
            
    \begin{Verbatim}[commandchars=\\\{\}]
{\color{incolor}In [{\color{incolor}27}]:} \PY{n}{vector\PYZus{}1}
\end{Verbatim}


\begin{Verbatim}[commandchars=\\\{\}]
{\color{outcolor}Out[{\color{outcolor}27}]:} Vector\_2D(3.0, 2.0)
\end{Verbatim}
            
    Another way to pass a value to a fieldname in a type is the
\texttt{setfield()} function. We have to use the correct type for the
value. If we use an integer such as
\texttt{setfield!(vector\_1,\ :x,\ 4)} we would get the follwoing error:

\begin{verbatim}
LoadError: TypeError: setfield!: expected Float64, got Int64
while loading In[25], in expression starting on line 1
\end{verbatim}

    \begin{Verbatim}[commandchars=\\\{\}]
{\color{incolor}In [{\color{incolor}28}]:} \PY{c}{\PYZsh{} Now we have to use a floating point value}
         \PY{n}{setfield!}\PY{p}{(}\PY{n}{vector\PYZus{}1}\PY{p}{,} \PY{o}{:}\PY{n}{x}\PY{p}{,} \PY{l+m+mf}{4.0}\PY{p}{)}
\end{Verbatim}


\begin{Verbatim}[commandchars=\\\{\}]
{\color{outcolor}Out[{\color{outcolor}28}]:} 4.0
\end{Verbatim}
            
    \begin{Verbatim}[commandchars=\\\{\}]
{\color{incolor}In [{\color{incolor}29}]:} \PY{c}{\PYZsh{} vector\PYZus{}1 has been changed}
         \PY{n}{vector\PYZus{}1}
\end{Verbatim}


\begin{Verbatim}[commandchars=\\\{\}]
{\color{outcolor}Out[{\color{outcolor}29}]:} Vector\_2D(4.0, 2.0)
\end{Verbatim}
            
    Section \ref{in-this-lesson}

    Convertion and promotion

    Before we go any further, we must have a look behind the scenes. Above
we saw that an integer was converted to a floating point value as
specified for the fields of our new type.

    \begin{Verbatim}[commandchars=\\\{\}]
{\color{incolor}In [{\color{incolor}30}]:} \PY{c}{\PYZsh{} Using the convert function}
         \PY{n}{convert}\PY{p}{(}\PY{k+kt}{Float64}\PY{p}{,} \PY{l+m+mi}{10}\PY{p}{)}
\end{Verbatim}


\begin{Verbatim}[commandchars=\\\{\}]
{\color{outcolor}Out[{\color{outcolor}30}]:} 10.0
\end{Verbatim}
            
    If precision is lost, convertion will result in an error. For instance,
\texttt{convert(Int16,\ 10.1)} will return:

\begin{verbatim}
adError: InexactError()
while loading In[20], in expression starting on line 1
\end{verbatim}

Using \texttt{convert(Int16,\ 10.0)} will return a value of \$ 10 \$,
though.

    Julia has a type promotion system that will try to incorporate values
into a single type. If we pass an integer and a floating point value,
the integer will be promoted to a floating point value.

    \begin{Verbatim}[commandchars=\\\{\}]
{\color{incolor}In [{\color{incolor}31}]:} \PY{n}{promote}\PY{p}{(}\PY{l+m+mi}{10}\PY{p}{,} \PY{l+m+mf}{10.0}\PY{p}{)}
\end{Verbatim}


\begin{Verbatim}[commandchars=\\\{\}]
{\color{outcolor}Out[{\color{outcolor}31}]:} (10.0, 10.0)
\end{Verbatim}
            
    \begin{Verbatim}[commandchars=\\\{\}]
{\color{incolor}In [{\color{incolor}32}]:} \PY{n}{typeof}\PY{p}{(}\PY{l+m+mi}{10}\PY{p}{)}
\end{Verbatim}


\begin{Verbatim}[commandchars=\\\{\}]
{\color{outcolor}Out[{\color{outcolor}32}]:} Int64
\end{Verbatim}
            
    \begin{Verbatim}[commandchars=\\\{\}]
{\color{incolor}In [{\color{incolor}33}]:} \PY{n}{typeof}\PY{p}{(}\PY{l+m+mf}{10.0}\PY{p}{)}
\end{Verbatim}


\begin{Verbatim}[commandchars=\\\{\}]
{\color{outcolor}Out[{\color{outcolor}33}]:} Float64
\end{Verbatim}
            
    This promotion to a common type lifts the lid on multiple dispatch when
a function is called with unspecified argument types (i.e.
\texttt{Any}).

    Section \ref{in-this-lesson}

    Parametrizing a type

    When creating a user type, we need not specify the type explicitely. We
could use a parameter. Have a look at the example below.

    \begin{Verbatim}[commandchars=\\\{\}]
{\color{incolor}In [{\color{incolor}34}]:} \PY{k}{type} \PY{n}{Vector\PYZus{}3D}\PY{p}{\PYZob{}}\PY{n}{T}\PY{p}{\PYZcb{}}
             \PY{n}{x}\PY{o}{::}\PY{n}{T}
             \PY{n}{y}\PY{o}{::}\PY{n}{T}
             \PY{n}{z}\PY{o}{::}\PY{n}{T}
         \PY{k}{end}
\end{Verbatim}


    We use \$ T \$ as a parameter placeholder. When we instantiate the type
we can use any appropriate type, as long as all the fields values are of
the same type.

    \begin{Verbatim}[commandchars=\\\{\}]
{\color{incolor}In [{\color{incolor}35}]:} \PY{c}{\PYZsh{} Using 64 bit integers}
         \PY{n}{vector\PYZus{}2} \PY{o}{=} \PY{n}{Vector\PYZus{}3D}\PY{p}{(}\PY{l+m+mi}{10}\PY{p}{,} \PY{l+m+mi}{12}\PY{p}{,} \PY{l+m+mi}{8}\PY{p}{)}
\end{Verbatim}


\begin{Verbatim}[commandchars=\\\{\}]
{\color{outcolor}Out[{\color{outcolor}35}]:} Vector\_3D\{Int64\}(10, 12, 8)
\end{Verbatim}
            
    If we were to execute \texttt{vector\_2\ =\ Vector\_3D(10.1,\ 10,\ 8)},
we would get the following error:

\begin{verbatim}
LoadError: MethodError: `convert` has no method matching convert(::Type{Vector_3D{T}}, ::Float64, ::Int64, ::Int64)
This may have arisen from a call to the constructor Vector_3D{T}(...),
since type constructors fall back to convert methods.
Closest candidates are:
  Vector_3D{T}(::T, !Matched::T, !Matched::T)
  call{T}(::Type{T}, ::Any)
  convert{T}(::Type{T}, !Matched::T)
while loading In[28], in expression starting on line 1
\end{verbatim}

    We can constrain the parametric type. Below we allow all subtypes of of
the abstract type \texttt{Real}.

    \begin{Verbatim}[commandchars=\\\{\}]
{\color{incolor}In [{\color{incolor}36}]:} \PY{k}{type} \PY{n}{Vector\PYZus{}3D\PYZus{}Real}\PY{p}{\PYZob{}}\PY{n}{T} \PY{o}{\PYZlt{}:} \PY{k+kt}{Real}\PY{p}{\PYZcb{}}
             \PY{n}{x}\PY{o}{::}\PY{n}{T}
             \PY{n}{y}\PY{o}{::}\PY{n}{T}
             \PY{n}{z}\PY{o}{::}\PY{n}{T}
         \PY{k}{end}
\end{Verbatim}


    As an aside, we cannot redefine a type. If we would use:

\begin{verbatim}
type Vector_3D{T}
    x::T
    y::T
    z::T
end
\end{verbatim}

we would get the error:

\begin{verbatim}
LoadError: invalid redefinition of constant Vector_3D
while loading In[98], in expression starting on line 1
\end{verbatim}

    \begin{Verbatim}[commandchars=\\\{\}]
{\color{incolor}In [{\color{incolor}37}]:} \PY{c}{\PYZsh{} Creating a new instance}
         \PY{n}{vector\PYZus{}3} \PY{o}{=} \PY{n}{Vector\PYZus{}3D\PYZus{}Real}\PY{p}{(}\PY{l+m+mi}{3}\PY{p}{,} \PY{l+m+mi}{3}\PY{p}{,} \PY{l+m+mi}{3}\PY{p}{)}
\end{Verbatim}


\begin{Verbatim}[commandchars=\\\{\}]
{\color{outcolor}Out[{\color{outcolor}37}]:} Vector\_3D\_Real\{Int64\}(3, 3, 3)
\end{Verbatim}
            
    Section \ref{in-this-lesson}

    The equality of values

    When are two values equal? We use a double equal sign to return a
Boolean value.

    \begin{Verbatim}[commandchars=\\\{\}]
{\color{incolor}In [{\color{incolor}38}]:} \PY{c}{\PYZsh{} Using the functional notation}
         \PY{o}{==}\PY{p}{(}\PY{l+m+mi}{5}\PY{p}{,} \PY{l+m+mf}{5.0}\PY{p}{)}
\end{Verbatim}


\begin{Verbatim}[commandchars=\\\{\}]
{\color{outcolor}Out[{\color{outcolor}38}]:} true
\end{Verbatim}
            
    Numbers are immutable and are compared at the bit level. This includes
their types. We can use the \texttt{===} sign or the \texttt{is()}
function to check for equality.

    \begin{Verbatim}[commandchars=\\\{\}]
{\color{incolor}In [{\color{incolor}41}]:} \PY{c}{\PYZsh{} New in 0.6}
         \PY{o}{===}\PY{p}{(}\PY{l+m+mi}{5}\PY{p}{,} \PY{l+m+mf}{5.0}\PY{p}{)}
\end{Verbatim}


\begin{Verbatim}[commandchars=\\\{\}]
{\color{outcolor}Out[{\color{outcolor}41}]:} false
\end{Verbatim}
            
    Where does this leave our user-defined types? We will see below that the
address in memory is checked when dealing with more complex objects such
as our user-defined, composite types.

    \begin{Verbatim}[commandchars=\\\{\}]
{\color{incolor}In [{\color{incolor}42}]:} \PY{n}{vector\PYZus{}a} \PY{o}{=} \PY{n}{Vector\PYZus{}2D}\PY{p}{(}\PY{l+m+mf}{1.0}\PY{p}{,} \PY{l+m+mf}{1.0}\PY{p}{)}
         \PY{n}{vector\PYZus{}b} \PY{o}{=} \PY{n}{Vector\PYZus{}2D}\PY{p}{(}\PY{l+m+mf}{1.0}\PY{p}{,} \PY{l+m+mf}{1.0}\PY{p}{)}
\end{Verbatim}


\begin{Verbatim}[commandchars=\\\{\}]
{\color{outcolor}Out[{\color{outcolor}42}]:} Vector\_2D(1.0, 1.0)
\end{Verbatim}
            
    \begin{Verbatim}[commandchars=\\\{\}]
{\color{incolor}In [{\color{incolor}44}]:} \PY{c}{\PYZsh{} New in 0.6}
         \PY{o}{===}\PY{p}{(}\PY{n}{vector\PYZus{}a}\PY{p}{,} \PY{n}{vector\PYZus{}b}\PY{p}{)}
\end{Verbatim}


\begin{Verbatim}[commandchars=\\\{\}]
{\color{outcolor}Out[{\color{outcolor}44}]:} false
\end{Verbatim}
            
    Section \ref{in-this-lesson}

    Defining methods for functions that will use user-types

    The summation function, \texttt{+()} has methods for adding different
types. What if we want to add two instances of our \texttt{Vector\_2D}
user-type? If we were to add \texttt{vector\_a} to \texttt{vector\_b} we
would get the following error:

\begin{verbatim}
LoadError: MethodError: `+` has no method matching +(::Vector_2D, ::Vector_2D)
Closest candidates are:
  +(::Any, ::Any, !Matched::Any, !Matched::Any...)
\end{verbatim}

    \begin{Verbatim}[commandchars=\\\{\}]
{\color{incolor}In [{\color{incolor}45}]:} \PY{c}{\PYZsh{} Base methods for the +() function}
         \PY{n}{methods}\PY{p}{(}\PY{o}{+}\PY{p}{)}
\end{Verbatim}


\begin{Verbatim}[commandchars=\\\{\}]
{\color{outcolor}Out[{\color{outcolor}45}]:} \# 180 methods for generic function "+":
         +(x::Bool, z::Complex\{Bool\}) in Base at complex.jl:232
         +(x::Bool, y::Bool) in Base at bool.jl:89
         +(x::Bool) in Base at bool.jl:86
         +(x::Bool, y::T) where T<:AbstractFloat in Base at bool.jl:96
         +(x::Bool, z::Complex) in Base at complex.jl:239
         +(a::Float16, b::Float16) in Base at float.jl:372
         +(x::Float32, y::Float32) in Base at float.jl:374
         +(x::Float64, y::Float64) in Base at float.jl:375
         +(z::Complex\{Bool\}, x::Bool) in Base at complex.jl:233
         +(z::Complex\{Bool\}, x::Real) in Base at complex.jl:247
         +(x::Char, y::Integer) in Base at char.jl:40
         +(c::BigInt, x::BigFloat) in Base.MPFR at mpfr.jl:312
         +(a::BigInt, b::BigInt, c::BigInt, d::BigInt, e::BigInt) in Base.GMP at gmp.jl:334
         +(a::BigInt, b::BigInt, c::BigInt, d::BigInt) in Base.GMP at gmp.jl:327
         +(a::BigInt, b::BigInt, c::BigInt) in Base.GMP at gmp.jl:321
         +(x::BigInt, y::BigInt) in Base.GMP at gmp.jl:289
         +(x::BigInt, c::Union\{UInt16, UInt32, UInt8\}) in Base.GMP at gmp.jl:346
         +(x::BigInt, c::Union\{Int16, Int32, Int8\}) in Base.GMP at gmp.jl:362
         +(a::BigFloat, b::BigFloat, c::BigFloat, d::BigFloat, e::BigFloat) in Base.MPFR at mpfr.jl:460
         +(a::BigFloat, b::BigFloat, c::BigFloat, d::BigFloat) in Base.MPFR at mpfr.jl:453
         +(a::BigFloat, b::BigFloat, c::BigFloat) in Base.MPFR at mpfr.jl:447
         +(x::BigFloat, c::BigInt) in Base.MPFR at mpfr.jl:308
         +(x::BigFloat, y::BigFloat) in Base.MPFR at mpfr.jl:277
         +(x::BigFloat, c::Union\{UInt16, UInt32, UInt8\}) in Base.MPFR at mpfr.jl:284
         +(x::BigFloat, c::Union\{Int16, Int32, Int8\}) in Base.MPFR at mpfr.jl:292
         +(x::BigFloat, c::Union\{Float16, Float32, Float64\}) in Base.MPFR at mpfr.jl:300
         +(B::BitArray\{2\}, J::UniformScaling) in Base.LinAlg at linalg\textbackslash{}uniformscaling.jl:59
         +(a::Base.Pkg.Resolve.VersionWeights.VWPreBuildItem, b::Base.Pkg.Resolve.VersionWeights.VWPreBuildItem) in Base.Pkg.Resolve.VersionWeights at pkg\textbackslash{}resolve\textbackslash{}versionweight.jl:87
         +(a::Base.Pkg.Resolve.VersionWeights.VWPreBuild, b::Base.Pkg.Resolve.VersionWeights.VWPreBuild) in Base.Pkg.Resolve.VersionWeights at pkg\textbackslash{}resolve\textbackslash{}versionweight.jl:135
         +(a::Base.Pkg.Resolve.VersionWeights.VersionWeight, b::Base.Pkg.Resolve.VersionWeights.VersionWeight) in Base.Pkg.Resolve.VersionWeights at pkg\textbackslash{}resolve\textbackslash{}versionweight.jl:197
         +(a::Base.Pkg.Resolve.MaxSum.FieldValues.FieldValue, b::Base.Pkg.Resolve.MaxSum.FieldValues.FieldValue) in Base.Pkg.Resolve.MaxSum.FieldValues at pkg\textbackslash{}resolve\textbackslash{}fieldvalue.jl:44
         +(x::Base.Dates.CompoundPeriod, y::Base.Dates.CompoundPeriod) in Base.Dates at dates\textbackslash{}periods.jl:349
         +(x::Base.Dates.CompoundPeriod, y::Base.Dates.Period) in Base.Dates at dates\textbackslash{}periods.jl:347
         +(x::Base.Dates.CompoundPeriod, y::Base.Dates.TimeType) in Base.Dates at dates\textbackslash{}periods.jl:387
         +(x::Date, y::Base.Dates.Day) in Base.Dates at dates\textbackslash{}arithmetic.jl:77
         +(x::Date, y::Base.Dates.Week) in Base.Dates at dates\textbackslash{}arithmetic.jl:75
         +(dt::Date, z::Base.Dates.Month) in Base.Dates at dates\textbackslash{}arithmetic.jl:58
         +(dt::Date, y::Base.Dates.Year) in Base.Dates at dates\textbackslash{}arithmetic.jl:32
         +(dt::Date, t::Base.Dates.Time) in Base.Dates at dates\textbackslash{}arithmetic.jl:20
         +(t::Base.Dates.Time, dt::Date) in Base.Dates at dates\textbackslash{}arithmetic.jl:24
         +(x::Base.Dates.Time, y::Base.Dates.TimePeriod) in Base.Dates at dates\textbackslash{}arithmetic.jl:81
         +(dt::DateTime, z::Base.Dates.Month) in Base.Dates at dates\textbackslash{}arithmetic.jl:52
         +(dt::DateTime, y::Base.Dates.Year) in Base.Dates at dates\textbackslash{}arithmetic.jl:28
         +(x::DateTime, y::Base.Dates.Period) in Base.Dates at dates\textbackslash{}arithmetic.jl:79
         +(y::AbstractFloat, x::Bool) in Base at bool.jl:98
         +(x::T, y::T) where T<:Union\{Int128, Int16, Int32, Int64, Int8, UInt128, UInt16, UInt32, UInt64, UInt8\} in Base at int.jl:32
         +(x::Integer, y::Ptr) in Base at pointer.jl:128
         +(z::Complex, w::Complex) in Base at complex.jl:221
         +(z::Complex, x::Bool) in Base at complex.jl:240
         +(x::Real, z::Complex\{Bool\}) in Base at complex.jl:246
         +(x::Real, z::Complex) in Base at complex.jl:258
         +(z::Complex, x::Real) in Base at complex.jl:259
         +(x::Rational, y::Rational) in Base at rational.jl:245
         +(x::Integer, y::Char) in Base at char.jl:41
         +(i::Integer, index::CartesianIndex) in Base.IteratorsMD at multidimensional.jl:79
         +(c::Union\{UInt16, UInt32, UInt8\}, x::BigInt) in Base.GMP at gmp.jl:350
         +(c::Union\{Int16, Int32, Int8\}, x::BigInt) in Base.GMP at gmp.jl:363
         +(c::Union\{UInt16, UInt32, UInt8\}, x::BigFloat) in Base.MPFR at mpfr.jl:288
         +(c::Union\{Int16, Int32, Int8\}, x::BigFloat) in Base.MPFR at mpfr.jl:296
         +(c::Union\{Float16, Float32, Float64\}, x::BigFloat) in Base.MPFR at mpfr.jl:304
         +(x::Irrational, y::Irrational) in Base at irrationals.jl:109
         +(x::Real, r::Base.Use\_StepRangeLen\_Instead) in Base at deprecated.jl:1223
         +(x::Number) in Base at operators.jl:399
         +(x::T, y::T) where T<:Number in Base at promotion.jl:335
         +(x::Number, y::Number) in Base at promotion.jl:249
         +(x::Real, r::AbstractUnitRange) in Base at range.jl:721
         +(x::Number, r::AbstractUnitRange) in Base at range.jl:723
         +(x::Number, r::StepRangeLen) in Base at range.jl:726
         +(x::Number, r::LinSpace) in Base at range.jl:730
         +(x::Number, r::Range) in Base at range.jl:724
         +(r::Range, x::Number) in Base at range.jl:732
         +(r1::OrdinalRange, r2::OrdinalRange) in Base at range.jl:882
         +(r1::LinSpace\{T\}, r2::LinSpace\{T\}) where T in Base at range.jl:889
         +(r1::StepRangeLen\{T,R,S\} where S, r2::StepRangeLen\{T,R,S\} where S) where \{R<:Base.TwicePrecision, T\} in Base at twiceprecision.jl:300
         +(r1::StepRangeLen\{T,S,S\} where S, r2::StepRangeLen\{T,S,S\} where S) where \{T, S\} in Base at range.jl:905
         +(r1::Union\{LinSpace, OrdinalRange, StepRangeLen\}, r2::Union\{LinSpace, OrdinalRange, StepRangeLen\}) in Base at range.jl:896
         +(x::Base.TwicePrecision, y::Number) in Base at twiceprecision.jl:454
         +(x::Number, y::Base.TwicePrecision) in Base at twiceprecision.jl:457
         +(x::Base.TwicePrecision\{T\}, y::Base.TwicePrecision\{T\}) where T in Base at twiceprecision.jl:460
         +(x::Base.TwicePrecision, y::Base.TwicePrecision) in Base at twiceprecision.jl:464
         +(x::Ptr, y::Integer) in Base at pointer.jl:126
         +(A::BitArray, B::BitArray) in Base at bitarray.jl:1176
         +(A::SymTridiagonal, B::SymTridiagonal) in Base.LinAlg at linalg\textbackslash{}tridiag.jl:128
         +(A::Tridiagonal, B::Tridiagonal) in Base.LinAlg at linalg\textbackslash{}tridiag.jl:624
         +(A::UpperTriangular, B::UpperTriangular) in Base.LinAlg at linalg\textbackslash{}triangular.jl:374
         +(A::LowerTriangular, B::LowerTriangular) in Base.LinAlg at linalg\textbackslash{}triangular.jl:375
         +(A::UpperTriangular, B::Base.LinAlg.UnitUpperTriangular) in Base.LinAlg at linalg\textbackslash{}triangular.jl:376
         +(A::LowerTriangular, B::Base.LinAlg.UnitLowerTriangular) in Base.LinAlg at linalg\textbackslash{}triangular.jl:377
         +(A::Base.LinAlg.UnitUpperTriangular, B::UpperTriangular) in Base.LinAlg at linalg\textbackslash{}triangular.jl:378
         +(A::Base.LinAlg.UnitLowerTriangular, B::LowerTriangular) in Base.LinAlg at linalg\textbackslash{}triangular.jl:379
         +(A::Base.LinAlg.UnitUpperTriangular, B::Base.LinAlg.UnitUpperTriangular) in Base.LinAlg at linalg\textbackslash{}triangular.jl:380
         +(A::Base.LinAlg.UnitLowerTriangular, B::Base.LinAlg.UnitLowerTriangular) in Base.LinAlg at linalg\textbackslash{}triangular.jl:381
         +(A::Base.LinAlg.AbstractTriangular, B::Base.LinAlg.AbstractTriangular) in Base.LinAlg at linalg\textbackslash{}triangular.jl:382
         +(A::Symmetric, x::Bool) in Base.LinAlg at linalg\textbackslash{}symmetric.jl:272
         +(A::Symmetric, x::Number) in Base.LinAlg at linalg\textbackslash{}symmetric.jl:274
         +(A::Hermitian, x::Bool) in Base.LinAlg at linalg\textbackslash{}symmetric.jl:272
         +(A::Hermitian, x::Real) in Base.LinAlg at linalg\textbackslash{}symmetric.jl:274
         +(Da::Diagonal, Db::Diagonal) in Base.LinAlg at linalg\textbackslash{}diagonal.jl:140
         +(A::Bidiagonal, B::Bidiagonal) in Base.LinAlg at linalg\textbackslash{}bidiag.jl:330
         +(UL::UpperTriangular, J::UniformScaling) in Base.LinAlg at linalg\textbackslash{}uniformscaling.jl:72
         +(UL::Base.LinAlg.UnitUpperTriangular, J::UniformScaling) in Base.LinAlg at linalg\textbackslash{}uniformscaling.jl:75
         +(UL::LowerTriangular, J::UniformScaling) in Base.LinAlg at linalg\textbackslash{}uniformscaling.jl:72
         +(UL::Base.LinAlg.UnitLowerTriangular, J::UniformScaling) in Base.LinAlg at linalg\textbackslash{}uniformscaling.jl:75
         +(A::Array, B::SparseMatrixCSC) in Base.SparseArrays at sparse\textbackslash{}sparsematrix.jl:1462
         +(x::Union\{Base.ReshapedArray\{T,1,A,MI\} where MI<:Tuple\{Vararg\{Base.MultiplicativeInverses.SignedMultiplicativeInverse\{Int64\},N\} where N\} where A<:DenseArray, DenseArray\{T,1\}, SubArray\{T,1,A,I,L\} where L\} where I<:Tuple\{Vararg\{Union\{Base.AbstractCartesianIndex, Int64, Range\{Int64\}\},N\} where N\} where A<:Union\{Base.ReshapedArray\{T,N,A,MI\} where MI<:Tuple\{Vararg\{Base.MultiplicativeInverses.SignedMultiplicativeInverse\{Int64\},N\} where N\} where A<:DenseArray where N where T, DenseArray\} where T, y::AbstractSparseArray\{Tv,Ti,1\} where Ti where Tv) in Base.SparseArrays at sparse\textbackslash{}sparsevector.jl:1333
         +(x::Union\{Base.ReshapedArray\{\#s267,N,A,MI\} where MI<:Tuple\{Vararg\{Base.MultiplicativeInverses.SignedMultiplicativeInverse\{Int64\},N\} where N\} where A<:DenseArray, DenseArray\{\#s267,N\}, SubArray\{\#s267,N,A,I,L\} where L\} where I<:Tuple\{Vararg\{Union\{Base.AbstractCartesianIndex, Int64, Range\{Int64\}\},N\} where N\} where A<:Union\{Base.ReshapedArray\{T,N,A,MI\} where MI<:Tuple\{Vararg\{Base.MultiplicativeInverses.SignedMultiplicativeInverse\{Int64\},N\} where N\} where A<:DenseArray where N where T, DenseArray\} where N where \#s267<:Union\{Base.Dates.CompoundPeriod, Base.Dates.Period\}) in Base.Dates at dates\textbackslash{}periods.jl:358
         +(A::SparseMatrixCSC, J::UniformScaling) in Base.SparseArrays at sparse\textbackslash{}sparsematrix.jl:3512
         +(A::AbstractArray\{TA,2\}, J::UniformScaling\{TJ\}) where \{TA, TJ\} in Base.LinAlg at linalg\textbackslash{}uniformscaling.jl:119
         +(A::Diagonal, B::Bidiagonal) in Base.LinAlg at linalg\textbackslash{}special.jl:113
         +(A::Bidiagonal, B::Diagonal) in Base.LinAlg at linalg\textbackslash{}special.jl:114
         +(A::Diagonal, B::Tridiagonal) in Base.LinAlg at linalg\textbackslash{}special.jl:113
         +(A::Tridiagonal, B::Diagonal) in Base.LinAlg at linalg\textbackslash{}special.jl:114
         +(A::Diagonal, B::Array\{T,2\} where T) in Base.LinAlg at linalg\textbackslash{}special.jl:113
         +(A::Array\{T,2\} where T, B::Diagonal) in Base.LinAlg at linalg\textbackslash{}special.jl:114
         +(A::Bidiagonal, B::Tridiagonal) in Base.LinAlg at linalg\textbackslash{}special.jl:113
         +(A::Tridiagonal, B::Bidiagonal) in Base.LinAlg at linalg\textbackslash{}special.jl:114
         +(A::Bidiagonal, B::Array\{T,2\} where T) in Base.LinAlg at linalg\textbackslash{}special.jl:113
         +(A::Array\{T,2\} where T, B::Bidiagonal) in Base.LinAlg at linalg\textbackslash{}special.jl:114
         +(A::Tridiagonal, B::Array\{T,2\} where T) in Base.LinAlg at linalg\textbackslash{}special.jl:113
         +(A::Array\{T,2\} where T, B::Tridiagonal) in Base.LinAlg at linalg\textbackslash{}special.jl:114
         +(A::SymTridiagonal, B::Tridiagonal) in Base.LinAlg at linalg\textbackslash{}special.jl:122
         +(A::Tridiagonal, B::SymTridiagonal) in Base.LinAlg at linalg\textbackslash{}special.jl:123
         +(A::SymTridiagonal, B::Array\{T,2\} where T) in Base.LinAlg at linalg\textbackslash{}special.jl:122
         +(A::Array\{T,2\} where T, B::SymTridiagonal) in Base.LinAlg at linalg\textbackslash{}special.jl:123
         +(A::Diagonal, B::SymTridiagonal) in Base.LinAlg at linalg\textbackslash{}special.jl:131
         +(A::SymTridiagonal, B::Diagonal) in Base.LinAlg at linalg\textbackslash{}special.jl:132
         +(A::Bidiagonal, B::SymTridiagonal) in Base.LinAlg at linalg\textbackslash{}special.jl:131
         +(A::SymTridiagonal, B::Bidiagonal) in Base.LinAlg at linalg\textbackslash{}special.jl:132
         +(A::Diagonal, B::UpperTriangular) in Base.LinAlg at linalg\textbackslash{}special.jl:143
         +(A::UpperTriangular, B::Diagonal) in Base.LinAlg at linalg\textbackslash{}special.jl:144
         +(A::Diagonal, B::Base.LinAlg.UnitUpperTriangular) in Base.LinAlg at linalg\textbackslash{}special.jl:143
         +(A::Base.LinAlg.UnitUpperTriangular, B::Diagonal) in Base.LinAlg at linalg\textbackslash{}special.jl:144
         +(A::Diagonal, B::LowerTriangular) in Base.LinAlg at linalg\textbackslash{}special.jl:143
         +(A::LowerTriangular, B::Diagonal) in Base.LinAlg at linalg\textbackslash{}special.jl:144
         +(A::Diagonal, B::Base.LinAlg.UnitLowerTriangular) in Base.LinAlg at linalg\textbackslash{}special.jl:143
         +(A::Base.LinAlg.UnitLowerTriangular, B::Diagonal) in Base.LinAlg at linalg\textbackslash{}special.jl:144
         +(A::Base.LinAlg.AbstractTriangular, B::SymTridiagonal) in Base.LinAlg at linalg\textbackslash{}special.jl:150
         +(A::SymTridiagonal, B::Base.LinAlg.AbstractTriangular) in Base.LinAlg at linalg\textbackslash{}special.jl:151
         +(A::Base.LinAlg.AbstractTriangular, B::Tridiagonal) in Base.LinAlg at linalg\textbackslash{}special.jl:150
         +(A::Tridiagonal, B::Base.LinAlg.AbstractTriangular) in Base.LinAlg at linalg\textbackslash{}special.jl:151
         +(A::Base.LinAlg.AbstractTriangular, B::Bidiagonal) in Base.LinAlg at linalg\textbackslash{}special.jl:150
         +(A::Bidiagonal, B::Base.LinAlg.AbstractTriangular) in Base.LinAlg at linalg\textbackslash{}special.jl:151
         +(A::Base.LinAlg.AbstractTriangular, B::Array\{T,2\} where T) in Base.LinAlg at linalg\textbackslash{}special.jl:150
         +(A::Array\{T,2\} where T, B::Base.LinAlg.AbstractTriangular) in Base.LinAlg at linalg\textbackslash{}special.jl:151
         +(Y::Union\{Base.ReshapedArray\{\#s266,N,A,MI\} where MI<:Tuple\{Vararg\{Base.MultiplicativeInverses.SignedMultiplicativeInverse\{Int64\},N\} where N\} where A<:DenseArray, DenseArray\{\#s266,N\}, SubArray\{\#s266,N,A,I,L\} where L\} where I<:Tuple\{Vararg\{Union\{Base.AbstractCartesianIndex, Int64, Range\{Int64\}\},N\} where N\} where A<:Union\{Base.ReshapedArray\{T,N,A,MI\} where MI<:Tuple\{Vararg\{Base.MultiplicativeInverses.SignedMultiplicativeInverse\{Int64\},N\} where N\} where A<:DenseArray where N where T, DenseArray\} where N where \#s266<:Union\{Base.Dates.CompoundPeriod, Base.Dates.Period\}, x::Union\{Base.Dates.CompoundPeriod, Base.Dates.Period\}) in Base.Dates at dates\textbackslash{}periods.jl:363
         +(X::Union\{Base.ReshapedArray\{\#s265,N,A,MI\} where MI<:Tuple\{Vararg\{Base.MultiplicativeInverses.SignedMultiplicativeInverse\{Int64\},N\} where N\} where A<:DenseArray, DenseArray\{\#s265,N\}, SubArray\{\#s265,N,A,I,L\} where L\} where I<:Tuple\{Vararg\{Union\{Base.AbstractCartesianIndex, Int64, Range\{Int64\}\},N\} where N\} where A<:Union\{Base.ReshapedArray\{T,N,A,MI\} where MI<:Tuple\{Vararg\{Base.MultiplicativeInverses.SignedMultiplicativeInverse\{Int64\},N\} where N\} where A<:DenseArray where N where T, DenseArray\} where N where \#s265<:Union\{Base.Dates.CompoundPeriod, Base.Dates.Period\}, Y::Union\{Base.ReshapedArray\{\#s264,N,A,MI\} where MI<:Tuple\{Vararg\{Base.MultiplicativeInverses.SignedMultiplicativeInverse\{Int64\},N\} where N\} where A<:DenseArray, DenseArray\{\#s264,N\}, SubArray\{\#s264,N,A,I,L\} where L\} where I<:Tuple\{Vararg\{Union\{Base.AbstractCartesianIndex, Int64, Range\{Int64\}\},N\} where N\} where A<:Union\{Base.ReshapedArray\{T,N,A,MI\} where MI<:Tuple\{Vararg\{Base.MultiplicativeInverses.SignedMultiplicativeInverse\{Int64\},N\} where N\} where A<:DenseArray where N where T, DenseArray\} where N where \#s264<:Union\{Base.Dates.CompoundPeriod, Base.Dates.Period\}) in Base.Dates at dates\textbackslash{}periods.jl:364
         +(x::Union\{Base.ReshapedArray\{\#s267,N,A,MI\} where MI<:Tuple\{Vararg\{Base.MultiplicativeInverses.SignedMultiplicativeInverse\{Int64\},N\} where N\} where A<:DenseArray, DenseArray\{\#s267,N\}, SubArray\{\#s267,N,A,I,L\} where L\} where I<:Tuple\{Vararg\{Union\{Base.AbstractCartesianIndex, Int64, Range\{Int64\}\},N\} where N\} where A<:Union\{Base.ReshapedArray\{T,N,A,MI\} where MI<:Tuple\{Vararg\{Base.MultiplicativeInverses.SignedMultiplicativeInverse\{Int64\},N\} where N\} where A<:DenseArray where N where T, DenseArray\} where N where \#s267<:Union\{Base.Dates.CompoundPeriod, Base.Dates.Period\}, y::Base.Dates.TimeType) in Base.Dates at dates\textbackslash{}arithmetic.jl:86
         +(r::Range\{\#s267\} where \#s267<:Base.Dates.TimeType, x::Base.Dates.Period) in Base.Dates at dates\textbackslash{}ranges.jl:47
         +(A::SparseMatrixCSC, B::SparseMatrixCSC) in Base.SparseArrays at sparse\textbackslash{}sparsematrix.jl:1458
         +(A::SparseMatrixCSC, B::Array) in Base.SparseArrays at sparse\textbackslash{}sparsematrix.jl:1461
         +(x::AbstractSparseArray\{Tv,Ti,1\} where Ti where Tv, y::AbstractSparseArray\{Tv,Ti,1\} where Ti where Tv) in Base.SparseArrays at sparse\textbackslash{}sparsevector.jl:1332
         +(x::AbstractSparseArray\{Tv,Ti,1\} where Ti where Tv, y::Union\{Base.ReshapedArray\{T,1,A,MI\} where MI<:Tuple\{Vararg\{Base.MultiplicativeInverses.SignedMultiplicativeInverse\{Int64\},N\} where N\} where A<:DenseArray, DenseArray\{T,1\}, SubArray\{T,1,A,I,L\} where L\} where I<:Tuple\{Vararg\{Union\{Base.AbstractCartesianIndex, Int64, Range\{Int64\}\},N\} where N\} where A<:Union\{Base.ReshapedArray\{T,N,A,MI\} where MI<:Tuple\{Vararg\{Base.MultiplicativeInverses.SignedMultiplicativeInverse\{Int64\},N\} where N\} where A<:DenseArray where N where T, DenseArray\} where T) in Base.SparseArrays at sparse\textbackslash{}sparsevector.jl:1334
         +(x::AbstractArray\{\#s45,N\} where N where \#s45<:Number) in Base at abstractarraymath.jl:93
         +(A::AbstractArray, B::AbstractArray) in Base at arraymath.jl:37
         +(A::Number, B::AbstractArray) in Base at arraymath.jl:44
         +(A::AbstractArray, B::Number) in Base at arraymath.jl:47
         +(index1::CartesianIndex\{N\}, index2::CartesianIndex\{N\}) where N in Base.IteratorsMD at multidimensional.jl:70
         +(index::CartesianIndex\{N\}, i::Integer) where N in Base.IteratorsMD at multidimensional.jl:80
         +(J1::UniformScaling, J2::UniformScaling) in Base.LinAlg at linalg\textbackslash{}uniformscaling.jl:58
         +(J::UniformScaling, B::BitArray\{2\}) in Base.LinAlg at linalg\textbackslash{}uniformscaling.jl:60
         +(J::UniformScaling, A::AbstractArray\{T,2\} where T) in Base.LinAlg at linalg\textbackslash{}uniformscaling.jl:61
         +(a::Base.Pkg.Resolve.VersionWeights.HierarchicalValue\{T\}, b::Base.Pkg.Resolve.VersionWeights.HierarchicalValue\{T\}) where T in Base.Pkg.Resolve.VersionWeights at pkg\textbackslash{}resolve\textbackslash{}versionweight.jl:23
         +(x::P, y::P) where P<:Base.Dates.Period in Base.Dates at dates\textbackslash{}periods.jl:70
         +(x::Base.Dates.Period, y::Base.Dates.Period) in Base.Dates at dates\textbackslash{}periods.jl:346
         +(y::Base.Dates.Period, x::Base.Dates.CompoundPeriod) in Base.Dates at dates\textbackslash{}periods.jl:348
         +(x::Union\{Base.Dates.CompoundPeriod, Base.Dates.Period\}) in Base.Dates at dates\textbackslash{}periods.jl:357
         +(x::Union\{Base.Dates.CompoundPeriod, Base.Dates.Period\}, Y::Union\{Base.ReshapedArray\{\#s267,N,A,MI\} where MI<:Tuple\{Vararg\{Base.MultiplicativeInverses.SignedMultiplicativeInverse\{Int64\},N\} where N\} where A<:DenseArray, DenseArray\{\#s267,N\}, SubArray\{\#s267,N,A,I,L\} where L\} where I<:Tuple\{Vararg\{Union\{Base.AbstractCartesianIndex, Int64, Range\{Int64\}\},N\} where N\} where A<:Union\{Base.ReshapedArray\{T,N,A,MI\} where MI<:Tuple\{Vararg\{Base.MultiplicativeInverses.SignedMultiplicativeInverse\{Int64\},N\} where N\} where A<:DenseArray where N where T, DenseArray\} where N where \#s267<:Union\{Base.Dates.CompoundPeriod, Base.Dates.Period\}) in Base.Dates at dates\textbackslash{}periods.jl:362
         +(x::Base.Dates.TimeType) in Base.Dates at dates\textbackslash{}arithmetic.jl:8
         +(a::Base.Dates.TimeType, b::Base.Dates.Period, c::Base.Dates.Period) in Base.Dates at dates\textbackslash{}periods.jl:378
         +(a::Base.Dates.TimeType, b::Base.Dates.Period, c::Base.Dates.Period, d::Base.Dates.Period{\ldots}) in Base.Dates at dates\textbackslash{}periods.jl:379
         +(x::Base.Dates.TimeType, y::Base.Dates.CompoundPeriod) in Base.Dates at dates\textbackslash{}periods.jl:382
         +(x::Base.Dates.Instant) in Base.Dates at dates\textbackslash{}arithmetic.jl:4
         +(y::Base.Dates.Period, x::Base.Dates.TimeType) in Base.Dates at dates\textbackslash{}arithmetic.jl:83
         +(x::AbstractArray\{\#s267,N\} where N where \#s267<:Base.Dates.TimeType, y::Union\{Base.Dates.CompoundPeriod, Base.Dates.Period\}) in Base.Dates at dates\textbackslash{}arithmetic.jl:85
         +(x::Base.Dates.Period, r::Range\{\#s267\} where \#s267<:Base.Dates.TimeType) in Base.Dates at dates\textbackslash{}ranges.jl:46
         +(y::Union\{Base.Dates.CompoundPeriod, Base.Dates.Period\}, x::AbstractArray\{\#s267,N\} where N where \#s267<:Base.Dates.TimeType) in Base.Dates at dates\textbackslash{}arithmetic.jl:87
         +(y::Base.Dates.TimeType, x::Union\{Base.ReshapedArray\{\#s267,N,A,MI\} where MI<:Tuple\{Vararg\{Base.MultiplicativeInverses.SignedMultiplicativeInverse\{Int64\},N\} where N\} where A<:DenseArray, DenseArray\{\#s267,N\}, SubArray\{\#s267,N,A,I,L\} where L\} where I<:Tuple\{Vararg\{Union\{Base.AbstractCartesianIndex, Int64, Range\{Int64\}\},N\} where N\} where A<:Union\{Base.ReshapedArray\{T,N,A,MI\} where MI<:Tuple\{Vararg\{Base.MultiplicativeInverses.SignedMultiplicativeInverse\{Int64\},N\} where N\} where A<:DenseArray where N where T, DenseArray\} where N where \#s267<:Union\{Base.Dates.CompoundPeriod, Base.Dates.Period\}) in Base.Dates at dates\textbackslash{}arithmetic.jl:88
         +(J::UniformScaling, x::Number) in Base at deprecated.jl:56
         +(x::Number, J::UniformScaling) in Base at deprecated.jl:56
         +(a, b, c, xs{\ldots}) in Base at operators.jl:424
\end{Verbatim}
            
    We have to create a method.

    \begin{Verbatim}[commandchars=\\\{\}]
{\color{incolor}In [{\color{incolor}46}]:} \PY{k}{import} \PY{n}{Base}\PY{o}{.+}
\end{Verbatim}


    \begin{Verbatim}[commandchars=\\\{\}]
{\color{incolor}In [{\color{incolor}47}]:} \PY{o}{+}\PY{p}{(}\PY{n}{u}\PY{o}{::}\PY{n}{Vector\PYZus{}2D}\PY{p}{,} \PY{n}{v}\PY{o}{::}\PY{n}{Vector\PYZus{}2D}\PY{p}{)} \PY{o}{=} \PY{n}{Vector\PYZus{}2D}\PY{p}{(}\PY{n}{u}\PY{o}{.}\PY{n}{x} \PY{o}{+} \PY{n}{v}\PY{o}{.}\PY{n}{x}\PY{p}{,} \PY{n}{u}\PY{o}{.}\PY{n}{y} \PY{o}{+} \PY{n}{v}\PY{o}{.}\PY{n}{y}\PY{p}{)}
\end{Verbatim}


\begin{Verbatim}[commandchars=\\\{\}]
{\color{outcolor}Out[{\color{outcolor}47}]:} + (generic function with 181 methods)
\end{Verbatim}
            
    \begin{Verbatim}[commandchars=\\\{\}]
{\color{incolor}In [{\color{incolor}48}]:} \PY{o}{+}\PY{p}{(}\PY{n}{vector\PYZus{}a}\PY{p}{,} \PY{n}{vector\PYZus{}b}\PY{p}{)}
\end{Verbatim}


\begin{Verbatim}[commandchars=\\\{\}]
{\color{outcolor}Out[{\color{outcolor}48}]:} Vector\_2D(2.0, 2.0)
\end{Verbatim}
            
    Section \ref{in-this-lesson}

    Constraining field values

    We can well imagine needing to constain the values that a type can hold.
Below we create the Bloodpressure type with two fields that hold integer
values. They cannot be negative and the systolic blood pressure must be
higher than the diastolic blood pressure. We solve this problem by
creating an inner constructor.

    \begin{Verbatim}[commandchars=\\\{\}]
{\color{incolor}In [{\color{incolor}49}]:} \PY{k}{type} \PY{n}{BloodPressure}
             \PY{c}{\PYZsh{} Don\PYZsq{}t leave as Any}
             \PY{n}{systolic}\PY{o}{::}\PY{k+kt}{Int16}
             \PY{n}{diastolic}\PY{o}{::}\PY{k+kt}{Int16}
             \PY{k}{function} \PY{n}{BloodPressure}\PY{p}{(}\PY{n}{s}\PY{p}{,} \PY{n}{d}\PY{p}{)}
                 \PY{c}{\PYZsh{} Using short\PYZhy{}circuit evaluations \PYZam{}\PYZam{} and ||}
                 \PY{n}{s} \PY{o}{\PYZlt{}} \PY{l+m+mi}{0} \PY{o}{\PYZam{}\PYZam{}} \PY{n}{throw}\PY{p}{(}\PY{k+kt}{ArgumentError}\PY{p}{(}\PY{l+s}{\PYZdq{}}\PY{l+s}{N}\PY{l+s}{e}\PY{l+s}{g}\PY{l+s}{a}\PY{l+s}{t}\PY{l+s}{i}\PY{l+s}{v}\PY{l+s}{e}\PY{l+s}{ }\PY{l+s}{p}\PY{l+s}{r}\PY{l+s}{e}\PY{l+s}{s}\PY{l+s}{s}\PY{l+s}{u}\PY{l+s}{r}\PY{l+s}{e}\PY{l+s}{s}\PY{l+s}{ }\PY{l+s}{a}\PY{l+s}{r}\PY{l+s}{e}\PY{l+s}{ }\PY{l+s}{n}\PY{l+s}{o}\PY{l+s}{t}\PY{l+s}{ }\PY{l+s}{a}\PY{l+s}{l}\PY{l+s}{l}\PY{l+s}{o}\PY{l+s}{w}\PY{l+s}{e}\PY{l+s}{d}\PY{l+s}{!}\PY{l+s}{\PYZdq{}}\PY{p}{)}\PY{p}{)}
                 \PY{n}{s} \PY{o}{\PYZlt{}=} \PY{n}{d} \PY{o}{\PYZam{}\PYZam{}} \PY{n}{throw}\PY{p}{(}\PY{k+kt}{ArgumentError}\PY{p}{(}\PY{l+s}{\PYZdq{}}\PY{l+s}{T}\PY{l+s}{h}\PY{l+s}{e}\PY{l+s}{ }\PY{l+s}{s}\PY{l+s}{y}\PY{l+s}{s}\PY{l+s}{t}\PY{l+s}{o}\PY{l+s}{l}\PY{l+s}{i}\PY{l+s}{c}\PY{l+s}{ }\PY{l+s}{b}\PY{l+s}{l}\PY{l+s}{o}\PY{l+s}{o}\PY{l+s}{d}\PY{l+s}{ }\PY{l+s}{p}\PY{l+s}{r}\PY{l+s}{e}\PY{l+s}{s}\PY{l+s}{s}\PY{l+s}{u}\PY{l+s}{r}\PY{l+s}{e}\PY{l+s}{ }\PY{l+s}{m}\PY{l+s}{u}\PY{l+s}{s}\PY{l+s}{t}\PY{l+s}{ }\PY{l+s}{b}\PY{l+s}{e}\PY{l+s}{ }\PY{l+s}{h}\PY{l+s}{i}\PY{l+s}{g}\PY{l+s}{h}\PY{l+s}{e}\PY{l+s}{r}\PY{l+s}{ }\PY{l+s}{t}\PY{l+s}{h}\PY{l+s}{a}\PY{l+s}{n}\PY{l+s}{ }\PY{l+s}{t}\PY{l+s}{h}\PY{l+s}{e}\PY{l+s}{ }\PY{l+s}{d}\PY{l+s}{i}\PY{l+s}{a}\PY{l+s}{s}\PY{l+s}{t}\PY{l+s}{o}\PY{l+s}{l}\PY{l+s}{i}\PY{l+s}{c}\PY{l+s}{ }\PY{l+s}{b}\PY{l+s}{l}\PY{l+s}{o}\PY{l+s}{o}\PY{l+s}{d}\PY{l+s}{ }\PY{l+s}{p}\PY{l+s}{r}\PY{l+s}{e}\PY{l+s}{s}\PY{l+s}{s}\PY{l+s}{u}\PY{l+s}{r}\PY{l+s}{e}\PY{l+s}{!}\PY{l+s}{\PYZdq{}}\PY{p}{)}\PY{p}{)}
                 \PY{n}{isa}\PY{p}{(}\PY{n}{s}\PY{p}{,} \PY{k+kt}{Integer}\PY{p}{)} \PY{o}{||} \PY{n}{throw}\PY{p}{(}\PY{k+kt}{ArgumentError}\PY{p}{(}\PY{l+s}{\PYZdq{}}\PY{l+s}{O}\PY{l+s}{n}\PY{l+s}{l}\PY{l+s}{y}\PY{l+s}{ }\PY{l+s}{i}\PY{l+s}{n}\PY{l+s}{t}\PY{l+s}{e}\PY{l+s}{g}\PY{l+s}{e}\PY{l+s}{r}\PY{l+s}{ }\PY{l+s}{v}\PY{l+s}{a}\PY{l+s}{l}\PY{l+s}{u}\PY{l+s}{e}\PY{l+s}{s}\PY{l+s}{ }\PY{l+s}{a}\PY{l+s}{l}\PY{l+s}{l}\PY{l+s}{o}\PY{l+s}{w}\PY{l+s}{e}\PY{l+s}{d}\PY{l+s}{!}\PY{l+s}{\PYZdq{}}\PY{p}{)}\PY{p}{)}
                 \PY{n}{isa}\PY{p}{(}\PY{n}{d}\PY{p}{,} \PY{k+kt}{Integer}\PY{p}{)} \PY{o}{||} \PY{n}{throw}\PY{p}{(}\PY{k+kt}{ArgumentError}\PY{p}{(}\PY{l+s}{\PYZdq{}}\PY{l+s}{O}\PY{l+s}{n}\PY{l+s}{l}\PY{l+s}{y}\PY{l+s}{ }\PY{l+s}{i}\PY{l+s}{n}\PY{l+s}{t}\PY{l+s}{e}\PY{l+s}{g}\PY{l+s}{e}\PY{l+s}{r}\PY{l+s}{ }\PY{l+s}{v}\PY{l+s}{a}\PY{l+s}{l}\PY{l+s}{u}\PY{l+s}{e}\PY{l+s}{s}\PY{l+s}{ }\PY{l+s}{a}\PY{l+s}{l}\PY{l+s}{l}\PY{l+s}{o}\PY{l+s}{w}\PY{l+s}{e}\PY{l+s}{d}\PY{l+s}{!}\PY{l+s}{\PYZdq{}}\PY{p}{)}\PY{p}{)}
                 \PY{n}{new}\PY{p}{(}\PY{n}{s}\PY{p}{,} \PY{n}{d}\PY{p}{)}
             \PY{k}{end}
         \PY{k}{end}
\end{Verbatim}


    \begin{Verbatim}[commandchars=\\\{\}]
{\color{incolor}In [{\color{incolor}50}]:} \PY{n}{bp\PYZus{}1} \PY{o}{=} \PY{n}{BloodPressure}\PY{p}{(}\PY{l+m+mi}{120}\PY{p}{,} \PY{l+m+mi}{80}\PY{p}{)}
\end{Verbatim}


\begin{Verbatim}[commandchars=\\\{\}]
{\color{outcolor}Out[{\color{outcolor}50}]:} BloodPressure(120, 80)
\end{Verbatim}
            
    Using \texttt{bp\_2\ =\ BloodPressure(-1,\ 90)} will result in the
error:

\begin{verbatim}
LoadError: ArgumentError: Negative pressures are not allowed!
while loading In[32], in expression starting on line 1
\end{verbatim}

    Using \texttt{bp\_2\ =\ BloodPressure(80,\ 120)} will result in the
error:

\begin{verbatim}
LoadError: ArgumentError: The systolic blood pressure must be higher than the diastolic blood pressure
while loading In[56], in expression starting on line 1
\end{verbatim}

    Using \texttt{bp\_2\ =\ BloodPressure(120.0,\ 80)} will result in the
error:

\begin{verbatim}
LoadError: ArgumentError: Only integer values allowed!
while loading In[95], in expression starting on line 1
\end{verbatim}

    Beware. Using inner constructors with parametrized types can lead to
problems.

    \begin{Verbatim}[commandchars=\\\{\}]
{\color{incolor}In [{\color{incolor}51}]:} \PY{k}{type} \PY{n}{BloodPressureParametrized}\PY{p}{\PYZob{}}\PY{n}{T} \PY{o}{\PYZlt{}:} \PY{k+kt}{Real}\PY{p}{\PYZcb{}}
             \PY{c}{\PYZsh{} Don\PYZsq{}t leave as Any}
             \PY{n}{systolic}\PY{o}{::}\PY{n}{T}
             \PY{n}{diastolic}\PY{o}{::}\PY{n}{T}
             \PY{k}{function} \PY{n}{BloodPressureParametrized}\PY{p}{(}\PY{n}{s}\PY{p}{,} \PY{n}{d}\PY{p}{)}
                 \PY{n}{s} \PY{o}{\PYZlt{}} \PY{l+m+mi}{0} \PY{o}{\PYZam{}\PYZam{}} \PY{n}{throw}\PY{p}{(}\PY{k+kt}{ArgumentError}\PY{p}{(}\PY{l+s}{\PYZdq{}}\PY{l+s}{N}\PY{l+s}{e}\PY{l+s}{g}\PY{l+s}{a}\PY{l+s}{t}\PY{l+s}{i}\PY{l+s}{v}\PY{l+s}{e}\PY{l+s}{ }\PY{l+s}{p}\PY{l+s}{r}\PY{l+s}{e}\PY{l+s}{s}\PY{l+s}{s}\PY{l+s}{u}\PY{l+s}{r}\PY{l+s}{e}\PY{l+s}{s}\PY{l+s}{ }\PY{l+s}{a}\PY{l+s}{r}\PY{l+s}{e}\PY{l+s}{ }\PY{l+s}{n}\PY{l+s}{o}\PY{l+s}{t}\PY{l+s}{ }\PY{l+s}{a}\PY{l+s}{l}\PY{l+s}{l}\PY{l+s}{o}\PY{l+s}{w}\PY{l+s}{e}\PY{l+s}{d}\PY{l+s}{!}\PY{l+s}{\PYZdq{}}\PY{p}{)}\PY{p}{)}
                 \PY{n}{s} \PY{o}{\PYZlt{}=} \PY{n}{d} \PY{o}{\PYZam{}\PYZam{}} \PY{n}{throw}\PY{p}{(}\PY{k+kt}{ArgumentError}\PY{p}{(}\PY{l+s}{\PYZdq{}}\PY{l+s}{T}\PY{l+s}{h}\PY{l+s}{e}\PY{l+s}{ }\PY{l+s}{s}\PY{l+s}{y}\PY{l+s}{s}\PY{l+s}{t}\PY{l+s}{o}\PY{l+s}{l}\PY{l+s}{i}\PY{l+s}{c}\PY{l+s}{ }\PY{l+s}{b}\PY{l+s}{l}\PY{l+s}{o}\PY{l+s}{o}\PY{l+s}{d}\PY{l+s}{ }\PY{l+s}{p}\PY{l+s}{r}\PY{l+s}{e}\PY{l+s}{s}\PY{l+s}{s}\PY{l+s}{u}\PY{l+s}{r}\PY{l+s}{e}\PY{l+s}{ }\PY{l+s}{m}\PY{l+s}{u}\PY{l+s}{s}\PY{l+s}{t}\PY{l+s}{ }\PY{l+s}{b}\PY{l+s}{e}\PY{l+s}{ }\PY{l+s}{h}\PY{l+s}{i}\PY{l+s}{g}\PY{l+s}{h}\PY{l+s}{e}\PY{l+s}{r}\PY{l+s}{ }\PY{l+s}{t}\PY{l+s}{h}\PY{l+s}{a}\PY{l+s}{n}\PY{l+s}{ }\PY{l+s}{t}\PY{l+s}{h}\PY{l+s}{e}\PY{l+s}{ }\PY{l+s}{d}\PY{l+s}{i}\PY{l+s}{a}\PY{l+s}{s}\PY{l+s}{t}\PY{l+s}{o}\PY{l+s}{l}\PY{l+s}{i}\PY{l+s}{c}\PY{l+s}{ }\PY{l+s}{b}\PY{l+s}{l}\PY{l+s}{o}\PY{l+s}{o}\PY{l+s}{d}\PY{l+s}{ }\PY{l+s}{p}\PY{l+s}{r}\PY{l+s}{e}\PY{l+s}{s}\PY{l+s}{s}\PY{l+s}{u}\PY{l+s}{r}\PY{l+s}{e}\PY{l+s}{!}\PY{l+s}{\PYZdq{}}\PY{p}{)}\PY{p}{)}
                 \PY{n}{isa}\PY{p}{(}\PY{n}{s}\PY{p}{,} \PY{k+kt}{Integer}\PY{p}{)} \PY{o}{||} \PY{n}{throw}\PY{p}{(}\PY{k+kt}{ArgumentError}\PY{p}{(}\PY{l+s}{\PYZdq{}}\PY{l+s}{O}\PY{l+s}{n}\PY{l+s}{l}\PY{l+s}{y}\PY{l+s}{ }\PY{l+s}{i}\PY{l+s}{n}\PY{l+s}{t}\PY{l+s}{e}\PY{l+s}{g}\PY{l+s}{e}\PY{l+s}{r}\PY{l+s}{ }\PY{l+s}{v}\PY{l+s}{a}\PY{l+s}{l}\PY{l+s}{u}\PY{l+s}{e}\PY{l+s}{s}\PY{l+s}{ }\PY{l+s}{a}\PY{l+s}{l}\PY{l+s}{l}\PY{l+s}{o}\PY{l+s}{w}\PY{l+s}{e}\PY{l+s}{d}\PY{l+s}{!}\PY{l+s}{\PYZdq{}}\PY{p}{)}\PY{p}{)}
                 \PY{n}{isa}\PY{p}{(}\PY{n}{d}\PY{p}{,} \PY{k+kt}{Integer}\PY{p}{)} \PY{o}{||} \PY{n}{throw}\PY{p}{(}\PY{k+kt}{ArgumentError}\PY{p}{(}\PY{l+s}{\PYZdq{}}\PY{l+s}{O}\PY{l+s}{n}\PY{l+s}{l}\PY{l+s}{y}\PY{l+s}{ }\PY{l+s}{i}\PY{l+s}{n}\PY{l+s}{t}\PY{l+s}{e}\PY{l+s}{g}\PY{l+s}{e}\PY{l+s}{r}\PY{l+s}{ }\PY{l+s}{v}\PY{l+s}{a}\PY{l+s}{l}\PY{l+s}{u}\PY{l+s}{e}\PY{l+s}{s}\PY{l+s}{ }\PY{l+s}{a}\PY{l+s}{l}\PY{l+s}{l}\PY{l+s}{o}\PY{l+s}{w}\PY{l+s}{e}\PY{l+s}{d}\PY{l+s}{!}\PY{l+s}{\PYZdq{}}\PY{p}{)}\PY{p}{)}
                 \PY{n}{new}\PY{p}{(}\PY{n}{s}\PY{p}{,} \PY{n}{d}\PY{p}{)}
             \PY{k}{end}
         \PY{k}{end}
\end{Verbatim}


    \begin{Verbatim}[commandchars=\\\{\}]

WARNING: deprecated syntax "inner constructor BloodPressureParametrized({\ldots}) around In[51]:6".
Use "BloodPressureParametrized\{T\}({\ldots}) where T" instead.

    \end{Verbatim}

    Using \texttt{bp\_3\ =\ BloodPressureParametrized(120,\ 80)} will result
in the error; ``\texttt{LoadError:\ MethodError:}convert` has no method
matching
convert(::Type\{BloodPressureParametrized\{T\textless{}:Real\}\},
::Int64, ::Int64) This may have arisen from a call to the constructor
BloodPressureParametrized\{T\textless{}:Real\}(...), since type
constructors fall back to convert methods. Closest candidates are:
call\{T\}(::Type\{T\}, ::Any) convert\{T\}(::Type\{T\}, !Matched::T)
while loading In{[}102{]}, in expression starting on line 1

in call at essentials.jl:57 ```

    Now we have to specify the type during the instantiation.

    \begin{Verbatim}[commandchars=\\\{\}]
{\color{incolor}In [{\color{incolor}52}]:} \PY{n}{bp\PYZus{}3} \PY{o}{=} \PY{n}{BloodPressureParametrized}\PY{p}{\PYZob{}}\PY{k+kt}{Int}\PY{p}{\PYZcb{}}\PY{p}{(}\PY{l+m+mi}{120}\PY{p}{,} \PY{l+m+mi}{80}\PY{p}{)}
\end{Verbatim}


\begin{Verbatim}[commandchars=\\\{\}]
{\color{outcolor}Out[{\color{outcolor}52}]:} BloodPressureParametrized\{Int64\}(120, 80)
\end{Verbatim}
            
    \begin{Verbatim}[commandchars=\\\{\}]
{\color{incolor}In [{\color{incolor}53}]:} \PY{k}{type} \PY{n}{BloodPressureParametrizedFixed}\PY{p}{\PYZob{}}\PY{n}{T} \PY{o}{\PYZlt{}:} \PY{k+kt}{Real}\PY{p}{\PYZcb{}}
             \PY{n}{systolic}\PY{o}{::}\PY{n}{T}
             \PY{n}{diastolic}\PY{o}{::}\PY{n}{T}
             \PY{k}{function} \PY{n}{BloodPressureParametrizedFixed}\PY{p}{(}\PY{n}{s}\PY{p}{,} \PY{n}{d}\PY{p}{)}
                 \PY{n}{s} \PY{o}{\PYZlt{}} \PY{l+m+mi}{0} \PY{o}{\PYZam{}\PYZam{}} \PY{n}{throw}\PY{p}{(}\PY{k+kt}{ArgumentError}\PY{p}{(}\PY{l+s}{\PYZdq{}}\PY{l+s}{N}\PY{l+s}{e}\PY{l+s}{g}\PY{l+s}{a}\PY{l+s}{t}\PY{l+s}{i}\PY{l+s}{v}\PY{l+s}{e}\PY{l+s}{ }\PY{l+s}{p}\PY{l+s}{r}\PY{l+s}{e}\PY{l+s}{s}\PY{l+s}{s}\PY{l+s}{u}\PY{l+s}{r}\PY{l+s}{e}\PY{l+s}{s}\PY{l+s}{ }\PY{l+s}{a}\PY{l+s}{r}\PY{l+s}{e}\PY{l+s}{ }\PY{l+s}{n}\PY{l+s}{o}\PY{l+s}{t}\PY{l+s}{ }\PY{l+s}{a}\PY{l+s}{l}\PY{l+s}{l}\PY{l+s}{o}\PY{l+s}{w}\PY{l+s}{e}\PY{l+s}{d}\PY{l+s}{!}\PY{l+s}{\PYZdq{}}\PY{p}{)}\PY{p}{)}
                 \PY{n}{s} \PY{o}{\PYZlt{}=} \PY{n}{d} \PY{o}{\PYZam{}\PYZam{}} \PY{n}{throw}\PY{p}{(}\PY{k+kt}{ArgumentError}\PY{p}{(}\PY{l+s}{\PYZdq{}}\PY{l+s}{T}\PY{l+s}{h}\PY{l+s}{e}\PY{l+s}{ }\PY{l+s}{s}\PY{l+s}{y}\PY{l+s}{s}\PY{l+s}{t}\PY{l+s}{o}\PY{l+s}{l}\PY{l+s}{i}\PY{l+s}{c}\PY{l+s}{ }\PY{l+s}{b}\PY{l+s}{l}\PY{l+s}{o}\PY{l+s}{o}\PY{l+s}{d}\PY{l+s}{ }\PY{l+s}{p}\PY{l+s}{r}\PY{l+s}{e}\PY{l+s}{s}\PY{l+s}{s}\PY{l+s}{u}\PY{l+s}{r}\PY{l+s}{e}\PY{l+s}{ }\PY{l+s}{m}\PY{l+s}{u}\PY{l+s}{s}\PY{l+s}{t}\PY{l+s}{ }\PY{l+s}{b}\PY{l+s}{e}\PY{l+s}{ }\PY{l+s}{h}\PY{l+s}{i}\PY{l+s}{g}\PY{l+s}{h}\PY{l+s}{e}\PY{l+s}{r}\PY{l+s}{ }\PY{l+s}{t}\PY{l+s}{h}\PY{l+s}{a}\PY{l+s}{n}\PY{l+s}{ }\PY{l+s}{t}\PY{l+s}{h}\PY{l+s}{e}\PY{l+s}{ }\PY{l+s}{d}\PY{l+s}{i}\PY{l+s}{a}\PY{l+s}{s}\PY{l+s}{t}\PY{l+s}{o}\PY{l+s}{l}\PY{l+s}{i}\PY{l+s}{c}\PY{l+s}{ }\PY{l+s}{b}\PY{l+s}{l}\PY{l+s}{o}\PY{l+s}{o}\PY{l+s}{d}\PY{l+s}{ }\PY{l+s}{p}\PY{l+s}{r}\PY{l+s}{e}\PY{l+s}{s}\PY{l+s}{s}\PY{l+s}{u}\PY{l+s}{r}\PY{l+s}{e}\PY{l+s}{!}\PY{l+s}{\PYZdq{}}\PY{p}{)}\PY{p}{)}
                 \PY{n}{new}\PY{p}{(}\PY{n}{s}\PY{p}{,} \PY{n}{d}\PY{p}{)}
             \PY{k}{end}
         \PY{k}{end}
\end{Verbatim}


    \begin{Verbatim}[commandchars=\\\{\}]

WARNING: deprecated syntax "inner constructor BloodPressureParametrizedFixed({\ldots}) around In[53]:5".
Use "BloodPressureParametrizedFixed\{T\}({\ldots}) where T" instead.

    \end{Verbatim}

    We can fix this by the assignment below.

    \begin{Verbatim}[commandchars=\\\{\}]
{\color{incolor}In [{\color{incolor}54}]:} \PY{c}{\PYZsh{} A bit of an effort}
         \PY{n}{BloodPressureParametrizedFixed}\PY{p}{\PYZob{}}\PY{n}{T}\PY{p}{\PYZcb{}}\PY{p}{(}\PY{n}{systolic}\PY{o}{::}\PY{n}{T}\PY{p}{,} \PY{n}{diastolic}\PY{o}{::}\PY{n}{T}\PY{p}{)} \PY{o}{=} \PY{n}{BloodPressureParametrizedFixed}\PY{p}{\PYZob{}}\PY{n}{T}\PY{p}{\PYZcb{}}\PY{p}{(}\PY{n}{systolic}\PY{p}{,} \PY{n}{diastolic}\PY{p}{)}
\end{Verbatim}


\begin{Verbatim}[commandchars=\\\{\}]
{\color{outcolor}Out[{\color{outcolor}54}]:} BloodPressureParametrizedFixed
\end{Verbatim}
            
    \begin{Verbatim}[commandchars=\\\{\}]
{\color{incolor}In [{\color{incolor}55}]:} \PY{n}{bp\PYZus{}4} \PY{o}{=} \PY{n}{BloodPressureParametrizedFixed}\PY{p}{(}\PY{l+m+mi}{120}\PY{p}{,} \PY{l+m+mi}{80}\PY{p}{)}
\end{Verbatim}


\begin{Verbatim}[commandchars=\\\{\}]
{\color{outcolor}Out[{\color{outcolor}55}]:} BloodPressureParametrizedFixed\{Int64\}(120, 80)
\end{Verbatim}
            
    We can get way more specific. In the code below we tell Julia that if we
pass integers to the type, they should be expressed as floating point
values.

    \begin{Verbatim}[commandchars=\\\{\}]
{\color{incolor}In [{\color{incolor}56}]:} \PY{n}{BloodPressureParametrizedFixed}\PY{p}{\PYZob{}}\PY{n}{T} \PY{o}{\PYZlt{}:} \PY{k+kt}{Int}\PY{p}{\PYZcb{}}\PY{p}{(}\PY{n}{systolic}\PY{o}{::}\PY{n}{T}\PY{p}{,} \PY{n}{diastolic}\PY{o}{::}\PY{n}{T}\PY{p}{)} \PY{o}{=} \PY{n}{BloodPressureParametrizedFixed}\PY{p}{\PYZob{}}\PY{k+kt}{Float64}\PY{p}{\PYZcb{}}\PY{p}{(}\PY{n}{systolic}\PY{p}{,} \PY{n}{diastolic}\PY{p}{)}
\end{Verbatim}


\begin{Verbatim}[commandchars=\\\{\}]
{\color{outcolor}Out[{\color{outcolor}56}]:} BloodPressureParametrizedFixed
\end{Verbatim}
            
    \begin{Verbatim}[commandchars=\\\{\}]
{\color{incolor}In [{\color{incolor}57}]:} \PY{n}{bp\PYZus{}5} \PY{o}{=} \PY{n}{BloodPressureParametrizedFixed}\PY{p}{(}\PY{l+m+mi}{120}\PY{p}{,} \PY{l+m+mi}{80}\PY{p}{)}
\end{Verbatim}


\begin{Verbatim}[commandchars=\\\{\}]
{\color{outcolor}Out[{\color{outcolor}57}]:} BloodPressureParametrizedFixed\{Float64\}(120.0, 80.0)
\end{Verbatim}
            
    Section \ref{in-this-lesson}

    More complex parameters

    Up until now we have constrained ourselves to a single parameter. It is
possible, though, to create more than one. Below we create a type called
\texttt{Relook}. It has one fieldname called \texttt{duration}, which
must be of subtype, \texttt{Real}. There is also a second parameter.

    \begin{Verbatim}[commandchars=\\\{\}]
{\color{incolor}In [{\color{incolor}58}]:} \PY{k}{type} \PY{n}{Relook}\PY{p}{\PYZob{}}\PY{n}{N}\PY{p}{,} \PY{n}{T}\PY{o}{\PYZlt{}:}\PY{k+kt}{Real}\PY{p}{\PYZcb{}}
             \PY{n}{duration}\PY{o}{::}\PY{n}{T}
         \PY{k}{end}
\end{Verbatim}


    Using \texttt{patient\_1\ =\ Relook(3,\ 60)} will result in the error:

\begin{verbatim}
LoadError: MethodError: `convert` has no method matching convert(::Type{Relook{N,T}}, ::Int64, ::Int64)
This may have arisen from a call to the constructor Relook{N,T}(...),
since type constructors fall back to convert methods.
Closest candidates are:
  call{T}(::Type{T}, ::Any)
  convert{T}(::Type{T}, !Matched::T)
while loading In[175], in expression starting on line 1

 in call at essentials.jl:57
\end{verbatim}

    \begin{Verbatim}[commandchars=\\\{\}]
{\color{incolor}In [{\color{incolor}59}]:} \PY{c}{\PYZsh{} We have to specify the type of the second parameter}
         \PY{n}{patient\PYZus{}1} \PY{o}{=} \PY{n}{Relook}\PY{p}{\PYZob{}}\PY{l+m+mi}{4}\PY{p}{,} \PY{k+kt}{Int16}\PY{p}{\PYZcb{}}\PY{p}{(}\PY{l+m+mi}{60}\PY{p}{)}
\end{Verbatim}


\begin{Verbatim}[commandchars=\\\{\}]
{\color{outcolor}Out[{\color{outcolor}59}]:} Relook\{4,Int16\}(60)
\end{Verbatim}
            
    \begin{Verbatim}[commandchars=\\\{\}]
{\color{incolor}In [{\color{incolor}60}]:} \PY{n}{patient\PYZus{}1}\PY{o}{.}\PY{n}{duration}
\end{Verbatim}


\begin{Verbatim}[commandchars=\\\{\}]
{\color{outcolor}Out[{\color{outcolor}60}]:} 60
\end{Verbatim}
            
    We now want to add only objects (instances) with the same value in the
first parameter.

    \begin{Verbatim}[commandchars=\\\{\}]
{\color{incolor}In [{\color{incolor}61}]:} \PY{o}{+}\PY{p}{\PYZob{}}\PY{n}{N}\PY{p}{,} \PY{n}{T}\PY{p}{\PYZcb{}}\PY{p}{(}\PY{n}{u}\PY{o}{::}\PY{n}{Relook}\PY{p}{\PYZob{}}\PY{n}{N}\PY{p}{,} \PY{n}{T}\PY{p}{\PYZcb{}}\PY{p}{,} \PY{n}{v}\PY{o}{::}\PY{n}{Relook}\PY{p}{\PYZob{}}\PY{n}{N}\PY{p}{,} \PY{n}{T}\PY{p}{\PYZcb{}}\PY{p}{)} \PY{o}{=} \PY{n}{Relook}\PY{p}{\PYZob{}}\PY{n}{N}\PY{p}{,} \PY{n}{T}\PY{p}{\PYZcb{}}\PY{p}{(}\PY{n}{u}\PY{o}{.}\PY{n}{duration} \PY{o}{+} \PY{n}{v}\PY{o}{.}\PY{n}{duration}\PY{p}{)}
\end{Verbatim}


\begin{Verbatim}[commandchars=\\\{\}]
{\color{outcolor}Out[{\color{outcolor}61}]:} + (generic function with 182 methods)
\end{Verbatim}
            
    \begin{Verbatim}[commandchars=\\\{\}]
{\color{incolor}In [{\color{incolor}62}]:} \PY{n}{patient\PYZus{}2} \PY{o}{=} \PY{n}{Relook}\PY{p}{\PYZob{}}\PY{l+m+mi}{4}\PY{p}{,} \PY{k+kt}{Int16}\PY{p}{\PYZcb{}}\PY{p}{(}\PY{l+m+mi}{70}\PY{p}{)}
\end{Verbatim}


\begin{Verbatim}[commandchars=\\\{\}]
{\color{outcolor}Out[{\color{outcolor}62}]:} Relook\{4,Int16\}(70)
\end{Verbatim}
            
    \begin{Verbatim}[commandchars=\\\{\}]
{\color{incolor}In [{\color{incolor}63}]:} \PY{n}{patient\PYZus{}1} \PY{o}{+} \PY{n}{patient\PYZus{}2}
\end{Verbatim}


\begin{Verbatim}[commandchars=\\\{\}]
{\color{outcolor}Out[{\color{outcolor}63}]:} Relook\{4,Int16\}(130)
\end{Verbatim}
            
    \begin{Verbatim}[commandchars=\\\{\}]
{\color{incolor}In [{\color{incolor}64}]:} \PY{n}{patient\PYZus{}3} \PY{o}{=} \PY{n}{Relook}\PY{p}{\PYZob{}}\PY{l+m+mi}{3}\PY{p}{,} \PY{k+kt}{Int16}\PY{p}{\PYZcb{}}\PY{p}{(}\PY{l+m+mi}{70}\PY{p}{)}
\end{Verbatim}


\begin{Verbatim}[commandchars=\\\{\}]
{\color{outcolor}Out[{\color{outcolor}64}]:} Relook\{3,Int16\}(70)
\end{Verbatim}
            
    Using \texttt{patient\_1\ +\ patient\_3} will now result in the error:
\texttt{LoadError:\ MethodError:\ `+`\ has\ no\ method\ matching\ +(::Relook\{4,Int16\},\ ::Relook\{3,Int16\})\ Closest\ candidates\ are:\ \ \ +(::Any,\ ::Any,\ !Matched::Any,\ !Matched::Any...)\ \ \ +\{N,T\}(::Relook\{N,T\},\ !Matched::Relook\{N,T\})\ while\ loading\ In{[}191{]},\ in\ expression\ starting\ on\ line\ 1}

    \begin{Verbatim}[commandchars=\\\{\}]
{\color{incolor}In [{\color{incolor}65}]:} \PY{n}{patient\PYZus{}4} \PY{o}{=} \PY{n}{Relook}\PY{p}{\PYZob{}}\PY{l+m+mf}{4.0}\PY{p}{,} \PY{k+kt}{Int16}\PY{p}{\PYZcb{}}\PY{p}{(}\PY{l+m+mi}{70}\PY{p}{)}
\end{Verbatim}


\begin{Verbatim}[commandchars=\\\{\}]
{\color{outcolor}Out[{\color{outcolor}65}]:} Relook\{4.0,Int16\}(70)
\end{Verbatim}
            
    Using \texttt{patient\_1\ +\ patient\_4} will also result in an error,
because the types of \texttt{N} do not match. The error would be:

\begin{verbatim}
LoadError: MethodError: `+` has no method matching +(::Relook{4,Int16}, ::Relook{4.0,Int16})
Closest candidates are:
  +(::Any, ::Any, !Matched::Any, !Matched::Any...)
  +{N,T}(::Relook{N,T}, !Matched::Relook{N,T})
  +{N,T1,T2}(::Relook{N,T1}, !Matched::Relook{N,T2})
while loading In[210], in expression starting on line 1
\end{verbatim}

    Below we fix the fieldname type mismatch.

    \begin{Verbatim}[commandchars=\\\{\}]
{\color{incolor}In [{\color{incolor}66}]:} \PY{o}{+}\PY{p}{\PYZob{}}\PY{n}{N}\PY{p}{,} \PY{n}{T1}\PY{p}{,} \PY{n}{T2}\PY{p}{\PYZcb{}}\PY{p}{(}\PY{n}{u}\PY{o}{::}\PY{n}{Relook}\PY{p}{\PYZob{}}\PY{n}{N}\PY{p}{,} \PY{n}{T1}\PY{p}{\PYZcb{}}\PY{p}{,} \PY{n}{v}\PY{o}{::}\PY{n}{Relook}\PY{p}{\PYZob{}}\PY{n}{N}\PY{p}{,} \PY{n}{T2}\PY{p}{\PYZcb{}}\PY{p}{)} \PY{o}{=} \PY{n}{Relook}\PY{p}{\PYZob{}}\PY{n}{N}\PY{p}{,} \PY{n}{promote\PYZus{}type}\PY{p}{(}\PY{n}{T1}\PY{p}{,} \PY{n}{T2}\PY{p}{)}\PY{p}{\PYZcb{}}\PY{p}{(}\PY{n}{u}\PY{o}{.}\PY{n}{duration} \PY{o}{+} \PY{n}{v}\PY{o}{.}\PY{n}{duration}\PY{p}{)}
\end{Verbatim}


\begin{Verbatim}[commandchars=\\\{\}]
{\color{outcolor}Out[{\color{outcolor}66}]:} + (generic function with 183 methods)
\end{Verbatim}
            
    \begin{Verbatim}[commandchars=\\\{\}]
{\color{incolor}In [{\color{incolor}67}]:} \PY{n}{patient\PYZus{}5} \PY{o}{=} \PY{n}{Relook}\PY{p}{\PYZob{}}\PY{l+m+mi}{4}\PY{p}{,} \PY{k+kt}{Float64}\PY{p}{\PYZcb{}}\PY{p}{(}\PY{l+m+mi}{60}\PY{p}{)}
\end{Verbatim}


\begin{Verbatim}[commandchars=\\\{\}]
{\color{outcolor}Out[{\color{outcolor}67}]:} Relook\{4,Float64\}(60.0)
\end{Verbatim}
            
    \begin{Verbatim}[commandchars=\\\{\}]
{\color{incolor}In [{\color{incolor}68}]:} \PY{c}{\PYZsh{} N = 4 for both patient\PYZus{}1 and patient\PYZus{}5}
         \PY{c}{\PYZsh{} T for patient\PYZus{}1 is Int16 and T for patient\PYZus{}2 is Float64}
         \PY{n}{patient\PYZus{}1} \PY{o}{+} \PY{n}{patient\PYZus{}5}
\end{Verbatim}


\begin{Verbatim}[commandchars=\\\{\}]
{\color{outcolor}Out[{\color{outcolor}68}]:} Relook\{4,Float64\}(120.0)
\end{Verbatim}
            
    If we want to throw an error if the number of relooks are not equal when
trying to add to obejcts of the \texttt{Relook} type, we can do the
following.

    \begin{Verbatim}[commandchars=\\\{\}]
{\color{incolor}In [{\color{incolor}69}]:} \PY{o}{+}\PY{p}{\PYZob{}}\PY{n}{N1}\PY{p}{,} \PY{n}{N2}\PY{p}{,} \PY{n}{T}\PY{p}{\PYZcb{}}\PY{p}{(}\PY{n}{u}\PY{o}{::}\PY{n}{Relook}\PY{p}{\PYZob{}}\PY{n}{N1}\PY{p}{,} \PY{n}{T}\PY{p}{\PYZcb{}}\PY{p}{,} \PY{n}{v}\PY{o}{::}\PY{n}{Relook}\PY{p}{\PYZob{}}\PY{n}{N2}\PY{p}{,} \PY{n}{T}\PY{p}{\PYZcb{}}\PY{p}{)} \PY{o}{=} 
         \PY{n}{throw}\PY{p}{(}\PY{k+kt}{ArgumentError}\PY{p}{(}\PY{l+s}{\PYZdq{}}\PY{l+s}{C}\PY{l+s}{a}\PY{l+s}{n}\PY{l+s}{n}\PY{l+s}{o}\PY{l+s}{t}\PY{l+s}{ }\PY{l+s}{a}\PY{l+s}{d}\PY{l+s}{d}\PY{l+s}{ }\PY{l+s}{d}\PY{l+s}{u}\PY{l+s}{r}\PY{l+s}{a}\PY{l+s}{t}\PY{l+s}{i}\PY{l+s}{o}\PY{l+s}{n}\PY{l+s}{s}\PY{l+s}{ }\PY{l+s}{w}\PY{l+s}{h}\PY{l+s}{e}\PY{l+s}{n}\PY{l+s}{ }\PY{l+s}{t}\PY{l+s}{h}\PY{l+s}{e}\PY{l+s}{ }\PY{l+s}{n}\PY{l+s}{u}\PY{l+s}{m}\PY{l+s}{b}\PY{l+s}{e}\PY{l+s}{r}\PY{l+s}{ }\PY{l+s}{o}\PY{l+s}{f}\PY{l+s}{ }\PY{l+s}{r}\PY{l+s}{e}\PY{l+s}{l}\PY{l+s}{o}\PY{l+s}{o}\PY{l+s}{k}\PY{l+s}{s}\PY{l+s}{ }\PY{l+s}{d}\PY{l+s}{o}\PY{l+s}{ }\PY{l+s}{n}\PY{l+s}{o}\PY{l+s}{t}\PY{l+s}{ }\PY{l+s}{m}\PY{l+s}{a}\PY{l+s}{t}\PY{l+s}{c}\PY{l+s}{h}\PY{l+s}{.}\PY{l+s}{\PYZdq{}}\PY{p}{)}\PY{p}{)}
\end{Verbatim}


\begin{Verbatim}[commandchars=\\\{\}]
{\color{outcolor}Out[{\color{outcolor}69}]:} + (generic function with 184 methods)
\end{Verbatim}
            
    Using \texttt{patient\_1\ +\ patient\_3} will now result in the error:

\begin{verbatim}
LoadError: ArgumentError: Cannot add durations when the number of relooks do not match.
while loading In[237], in expression starting on line 1

 in + at In[236]:1
\end{verbatim}

    What about calculating the natural logarithm of the duration of a
\texttt{Relook} object? We could just specify the field name.

    \begin{Verbatim}[commandchars=\\\{\}]
{\color{incolor}In [{\color{incolor}70}]:} \PY{n}{log}\PY{p}{(}\PY{n}{patient\PYZus{}1}\PY{o}{.}\PY{n}{duration}\PY{p}{)}
\end{Verbatim}


\begin{Verbatim}[commandchars=\\\{\}]
{\color{outcolor}Out[{\color{outcolor}70}]:} 4.0943445622221
\end{Verbatim}
            
    Better still, we could specify wat the \texttt{log} function actually
does with a \texttt{Relook} object.

    \begin{Verbatim}[commandchars=\\\{\}]
{\color{incolor}In [{\color{incolor}71}]:} \PY{k}{import} \PY{n}{Base}\PY{o}{.}\PY{n}{log}
\end{Verbatim}


    \begin{Verbatim}[commandchars=\\\{\}]
{\color{incolor}In [{\color{incolor}72}]:} \PY{n}{log}\PY{p}{(}\PY{n}{u}\PY{o}{::}\PY{n}{Relook}\PY{p}{)} \PY{o}{=} \PY{n}{log}\PY{p}{(}\PY{n}{u}\PY{o}{.}\PY{n}{duration}\PY{p}{)}
\end{Verbatim}


\begin{Verbatim}[commandchars=\\\{\}]
{\color{outcolor}Out[{\color{outcolor}72}]:} log (generic function with 19 methods)
\end{Verbatim}
            
    \begin{Verbatim}[commandchars=\\\{\}]
{\color{incolor}In [{\color{incolor}73}]:} \PY{n}{log}\PY{p}{(}\PY{n}{patient\PYZus{}1}\PY{p}{)}
\end{Verbatim}


\begin{Verbatim}[commandchars=\\\{\}]
{\color{outcolor}Out[{\color{outcolor}73}]:} 4.0943445622221
\end{Verbatim}
            
    We can specify a \texttt{convert} method that will convert all our
\texttt{Relook} objects to numerical values which we can pass to Julia
functions.

    \begin{Verbatim}[commandchars=\\\{\}]
{\color{incolor}In [{\color{incolor}74}]:} \PY{k}{import} \PY{n}{Base}\PY{o}{.}\PY{n}{convert}
\end{Verbatim}


    \begin{Verbatim}[commandchars=\\\{\}]
{\color{incolor}In [{\color{incolor}75}]:} \PY{c}{\PYZsh{} The float() function tries to convert a value to a floating point value}
         \PY{n}{convert}\PY{p}{(}\PY{o}{::}\PY{k+kt}{Type}\PY{p}{\PYZob{}}\PY{k+kt}{AbstractFloat}\PY{p}{\PYZcb{}}\PY{p}{,} \PY{n}{u}\PY{o}{::}\PY{n}{Relook}\PY{p}{)} \PY{o}{=} \PY{n}{float}\PY{p}{(}\PY{n}{u}\PY{o}{.}\PY{n}{duration}\PY{p}{)}
\end{Verbatim}


\begin{Verbatim}[commandchars=\\\{\}]
{\color{outcolor}Out[{\color{outcolor}75}]:} convert (generic function with 708 methods)
\end{Verbatim}
            
    \begin{Verbatim}[commandchars=\\\{\}]
{\color{incolor}In [{\color{incolor}76}]:} \PY{c}{\PYZsh{} Covreting our Relook object and passing it to log10()}
         \PY{n}{log10}\PY{p}{(}\PY{n}{convert}\PY{p}{(}\PY{k+kt}{AbstractFloat}\PY{p}{,} \PY{n}{patient\PYZus{}1}\PY{p}{)}\PY{p}{)}
\end{Verbatim}


\begin{Verbatim}[commandchars=\\\{\}]
{\color{outcolor}Out[{\color{outcolor}76}]:} 1.7781512503836436
\end{Verbatim}
            
    Section \ref{in-this-lesson}

    Screen output of a user-defined type

    We can create some meaning to our types by the way an object of the type
is represented on the screen. Above we just saw two values we
instantiation our type \texttt{Relook}. Let's change that a bit by
overloading the \texttt{show} function.

    \begin{Verbatim}[commandchars=\\\{\}]
{\color{incolor}In [{\color{incolor}77}]:} \PY{k}{import} \PY{n}{Base}\PY{o}{.}\PY{n}{show}
\end{Verbatim}


    \begin{Verbatim}[commandchars=\\\{\}]
{\color{incolor}In [{\color{incolor}78}]:} \PY{n}{show}\PY{p}{\PYZob{}}\PY{n}{N}\PY{p}{,} \PY{n}{T}\PY{p}{\PYZcb{}}\PY{p}{(}\PY{n}{io}\PY{o}{::}\PY{k+kt}{IO}\PY{p}{,} \PY{n}{u}\PY{o}{::}\PY{n}{Relook}\PY{p}{\PYZob{}}\PY{n}{N}\PY{p}{,} \PY{n}{T}\PY{p}{\PYZcb{}}\PY{p}{)} \PY{o}{=} \PY{n}{print}\PY{p}{(}\PY{n}{io}\PY{p}{,} \PY{l+s}{\PYZdq{}}\PY{l+s}{P}\PY{l+s}{a}\PY{l+s}{t}\PY{l+s}{i}\PY{l+s}{e}\PY{l+s}{n}\PY{l+s}{t}\PY{l+s}{ }\PY{l+s}{w}\PY{l+s}{i}\PY{l+s}{t}\PY{l+s}{h}\PY{l+s}{ }\PY{l+s}{\PYZdq{}}\PY{p}{,} \PY{n}{N}\PY{p}{,} \PY{l+s}{\PYZdq{}}\PY{l+s}{ }\PY{l+s}{r}\PY{l+s}{e}\PY{l+s}{l}\PY{l+s}{o}\PY{l+s}{o}\PY{l+s}{k}\PY{l+s}{ }\PY{l+s}{p}\PY{l+s}{r}\PY{l+s}{o}\PY{l+s}{c}\PY{l+s}{e}\PY{l+s}{d}\PY{l+s}{u}\PY{l+s}{r}\PY{l+s}{e}\PY{l+s}{s}\PY{l+s}{ }\PY{l+s}{t}\PY{l+s}{o}\PY{l+s}{t}\PY{l+s}{a}\PY{l+s}{l}\PY{l+s}{l}\PY{l+s}{i}\PY{l+s}{n}\PY{l+s}{g}\PY{l+s}{ }\PY{l+s}{\PYZdq{}}\PY{p}{,} \PY{n}{u}\PY{o}{.}\PY{n}{duration}\PY{p}{,}
         \PY{l+s}{\PYZdq{}}\PY{l+s}{ }\PY{l+s}{m}\PY{l+s}{i}\PY{l+s}{n}\PY{l+s}{u}\PY{l+s}{t}\PY{l+s}{e}\PY{l+s}{s}\PY{l+s}{.}\PY{l+s}{\PYZdq{}}\PY{p}{)}
\end{Verbatim}


\begin{Verbatim}[commandchars=\\\{\}]
{\color{outcolor}Out[{\color{outcolor}78}]:} show (generic function with 268 methods)
\end{Verbatim}
            
    \begin{Verbatim}[commandchars=\\\{\}]
{\color{incolor}In [{\color{incolor}79}]:} \PY{n}{patient\PYZus{}6} \PY{o}{=} \PY{n}{Relook}\PY{p}{\PYZob{}}\PY{l+m+mi}{4}\PY{p}{,} \PY{k+kt}{Int16}\PY{p}{\PYZcb{}}\PY{p}{(}\PY{l+m+mi}{60}\PY{p}{)}
\end{Verbatim}


\begin{Verbatim}[commandchars=\\\{\}]
{\color{outcolor}Out[{\color{outcolor}79}]:} Patient with 4 relook procedures totalling 60 minutes.
\end{Verbatim}
            
    Section \ref{in-this-lesson}


    % Add a bibliography block to the postdoc
    
    
    
    \end{document}
